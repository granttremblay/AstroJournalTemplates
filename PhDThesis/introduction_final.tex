
\chapter[Introduction]{{\LARGE{\sc Introduction}\\The Heating {\it of} Rapidly Cooling Cluster Cores}}
\vspace*{-0.5cm}
\epigraph{{\it And whether or not it is clear to you,\\
no doubt the Universe is unfolding as it should.\\
{\sc Desiderata, 1927}}}

\label{chapter:introduction}


\vspace*{-0.4cm} 

The hierarchical assembly of luminous structure in the
Universe begins with the first stars at the epoch of reionization and ends with
massive clusters of galaxies after a Hubble time.  Observations of the latter
suggest that the process is inefficient with respect to the formation of stars
and galaxies, as these comprise only ten percent of the baryonic mass fraction
in a galaxy cluster.  The majority of baryons do not participate in the growth
of structural complexity, and are instead heated at early epochs by adiabatic
compression and accretion shocks to tens of millions of kelvin, roughly the
virial temperature of the cold dark matter halos into which they collapse. The
result is a quasi-hydrostatic bath of X-ray bright, optically thin plasma
occupying hundreds of cubic megaparsecs.  The processes regulating the entropy
of this intracluster medium are poorly understood, but  surely fundamental to
our understanding of galaxy evolution as a whole.  This doctoral thesis 
presents new observational insights into these issues, for which 
we now provide a contextual summary.



\begin{figure*}
\begin{center}$
\begin{array}{cc}
\includegraphics[angle=0,width=3.1in]{springel_1.eps}  &
\includegraphics[angle=0,width=3.1in]{springel_2.eps}  \\
\includegraphics[angle=0,width=3.1in]{springel_3.eps}  &
\includegraphics[angle=0,width=3.1in]{springel_4.eps}  \\
\end{array}$
\end{center}
\caption[The hierarchical growth of cold dark matter halos throughout cosmic time]{Both dark matter halo and baryonic structure grows hierarchically in the $\Lambda$CDM cosmological paradigm. This visualization, from the 10$^{10}$ particle {\it Millennium} 
simulation of \citet{springel05}, shows the spatial density 
of dark matter halos at various epochs in the simulation. In the top left panel, the Universe is 0.2 Gyr old, while the bottom right panel shows the distribution of dark matter as it might exist now. 
The massive halo at the core of the filamentary web would be host to a rich cluster of galaxies. 
Each visualization is approximately 70 Mpc $h^{-1}$ across and projects a slice through the density field 15 Mpc  $h^{-1}$ thick.}
\label{fig:intro_springel}
\end{figure*}



\section{An Overview of the Science in this Thesis}


For  a subset  of galaxy  clusters\footnote{This section (1.1) provides a focused and brief summary of
 the background 
material that is directly relevant to this thesis. So as not to 
distract the reader with too much secondary or contextual detail, 
some prior knowledge of galaxy clusters is assumed. 
A longer, more general review of this material can be found 
in Section 1.2.} with  sharply peaked  X-ray surface
brightness     profiles,     the     intracluster     medium     (ICM,
e.g.,~\citealt{sarazin86}) can cool  via bremsstrahlung processes from
$>10^7$ K to $\ll 10^4$ K on timescales much shorter than a $\sim$Gyr within
a radius of $\sim$100 kpc\footnote{One parsec (pc) is approximately $3.26$ light years (ly)
or $\sim3.09\times10^{18}$ cm.}.  Simple models predict that runaway entropy
loss by  gas within this radius accompanies  subsonic, nearly isobaric
compression  by  the  ambient  hot  reservoir,  driving  a  long-lived
classical ``cooling  flow'' onto the central  brightest cluster galaxy
(hereafter  BCG, e.g.,  \citealt{lea73,cowie77,fabian77};  see cooling
flow  reviews by  \citealt{fabian94,peterson06}).   Such a  phenomenon
might  constitute   an  observable  low-redshift   analog  to  cooling
processes  thought to  drive the  formation and  evolution  of massive
galaxies at early epochs (e.g., \citealt{silk77,rees77}).  Amongst the
largest  galaxies in  the Universe,  BCGs in  rapidly  cooling cluster
cores  (``cool  cores'', hereafter  CC\footnote{ The
defining characteristic of a ``cool core'' cluster is somewhat inconsistent
throughout the literature.  CC clusters have been classified as such given (1)
an observed central temperature drop (e.g., \citealt{sanderson06,burns08}), (2)
a short {\it inferred} central cooling time $t_\mathrm{cool}$ (shorter than the
age of the cluster, i.e., $t_\mathrm{cool}\lae1/H_0$ or sometimes
$t_\mathrm{cool} < 10$ Gyr, e.g.,  \citealt{bauer05,donahue07}), or (3) a high
classical mass deposition rate (e.g., \citealt{chen07}). For a recent
discussion of the issue, see \citet{hudson10}.  This thesis exclusively focuses
on clusters whose CC status is unambiguous, regardless of the specific
criterion used (meaning they satisfy all three). See section
\ref{section:coolingflows} for a more detailed discussion of cool core clusters
and their historical association with the cooling flow model.})  therefore represent  critical
tests for hierarchical structure formation models.






Two decades of progress have  demonstrated that the cooling flow model
historically associated  with CC clusters  fails in the absence  of an
additional,   non-gravitational    heating   mechanism.   Uninhibited,
catastrophic condensation  should drive massive  cold gas repositories
($\sim10^{12}$  \Msol,  e.g.,  \citealt{fabian94,odea97}) and  extreme
star  formation rates in  the BCG  ($10^2-10^3$ \Msol\  yr\mone, e.g.,
\citealt{peres98}),  but results  from searches  for these  mass sinks
were     often    orders     of     magnitude    below     predictions
\citep[e.g.,][]{allen69,deyoung74,peterson78,haynes78,baan78,shostak80,chrisstefi87,mcnamara89,odea94,odea97,odea98,allen95,mittaz01,edge03}.
Moreover, the  absence or  strong under-production of  expected coolant
lines in  high resolution  X-ray spectroscopy of  large CC
cluster samples effectively proves  that no more than 10\%  
of the supposedly cooling   gas  
actually cools below $\lae  1$   keV  (e.g., Fig.~\ref{fig:intro_petersonspec} and
\citealt{tamura01,peterson01,peterson03,xu02,sakelliou02,sanders08}).
%, although
%Fe {\sc xvii} observations of the CC cluster Abell 2597 (the subject of this
%paper) are consistent with a cooling rate $>100$ \Msol\
%yr\mone\citep{morris05}. 





\begin{figure}
\plotone{abell1689.eps}
\caption[Optical and X-ray composite of the galaxy cluster Abell 1689]{{\it Hubble Space Telescope} (optical) and {\it Chandra X-ray Observatory} (X-ray, in purple) composite image of the galaxy cluster Abell 1689 ($z=0.18$, corresponding to a luminosity distance of order 2.8 billion light years). The hot ($10^{7}-10^{8}$ K or $1-10$ keV), X-ray luminous intracluster gas (in purple) comprises $\sim 80-90 \%$ of the cluster's {\it baryonic} mass fraction. The entire cluster resides within a dark matter halo comprising 90\% of the {\it total} mass fraction. Clusters of galaxies are the largest 
virialized structures in the Universe, containing masses exceeding $10^{15}$ \Msol\ within radii that can reach several Mpc. The faint arcs in the image arise from the gravitational lensing effect, wherein the massive cluster warps spacetime and alters the apparent trajectory of photons passing through it. The field of view of the image is $\sim 3.2\arcmin$, corresponding to a projected angular size of order $\sim 0.5$ Mpc. (Credit: X-ray: NASA/CXC/MIT/E.~H. Peng et al.; Optical: NASA/STScI). }
\label{fig:intro_a1689}
\end{figure}


\subsection{Brightest Cluster Galaxies in Cool Cores}


Nevertheless,  there  is  circumstantial  evidence that  some  residual cooling
to $\lae 10^4$ K
manages to persist at a  few percent of the expected levels.  Relative
to BCGs  in non-cool core  clusters or field giant  ellipticals (gEs),
BCGs  in cool  cores are  far more  likely to  harbor H$\alpha$-bright
optical emission  line nebulae with  complex filamentary morphologies,
blue  continuum excess  associated  with low  levels  of ongoing  star
formation  ($\lae  10$  \Msol\  yr\mone), and  compact  central  radio
sources (e.g., the archetypal NGC 1275 / Perseus, see Fig.~\ref{fig:intro_fabian} and e.g.,
\citealt{minkowski57,lynds70,sabra00,conselice01,fabian03}).  The power of these phenomena appears to correlate with upper
limits      on      X-ray      derived     ICM      cooling      rates
\citep{hu85,odea87,heckman89,mcnamara04,rafferty06,salome06,quillen08,odea08},
and the  emission line  nebulae are almost  always cospatial  with the
coolest  X-ray gas  \citep{crawford95,crawford05,fabian03}.
In a few cases, estimates of condensation and star formation rates 
are in near agreement \citep{odea08}. 
Moreover,
repositories  of  up to  $10^{10}-10^{11}$  \Msol\  of cold\footnote{Throughout this thesis,
we will use ``hot'' to describe $10^7 < T < 10^8$ K (X-ray bright) ICM/ISM
phases, ``warm'' to describe $\sim10^4 < T < 10^5$ K (optical and UV bright)
components, and ``cold'' to describe $\sim10<T<10^4$ K (N/M/FIR bright)
components. By nature of the data we will present, we will not discuss the
critically important $10^5-10^7$ K regime at great length in this thesis.}  molecular
hydrogen have been  detected on 20 kpc scales  through CO observations
in   $15-30$\%   of   CC    clusters   (depending   on   the   sample,
\citealt{edge01,edge03,salome06}),  as have extended  distributions of
vibrationally             excited            H$_2$            emission
\citep{donahue00,jaffe01,egami06,wilman09}.  Recently,  {\it  Herschel
  Space Observatory}  observations of 11  CC BCGs (hereafter  the {\it
  Herschel} OTKP  sample) have  confirmed the presence  of substantial
($\sim10^{10}$ \Msol) cold  gas reservoirs via spectroscopic detection
of  major   coolant  lines  (e.g.,   [\ion{C}{2}]$\lambda$157  $\mu$m,
[\ion{O}{1}]$\lambda$68   $\mu$m)   stemming    from   $<50$   K   gas
(\citealt{edge10spec,edge10phot}; Oonk et al.~2011, in preparation;   Mittal   et  al.~2011,   in
press). These pools of cold gas appear to be correlated with the luminosity, dynamics 
and morphology of the H$\alpha$ nebulae. 








\begin{figure}
\begin{center}
\includegraphics[scale=0.75]{peterson_spectroscopy_plot.ps}
\end{center}
\caption[One instance in which the uninhibited cooling flow model fails to match observations]{Presented here is one of the
most important instances in which the classical, uninhibited cooling flow model to fails match observations. Plotted in blue is the high spectral resolution 
{\it XMM-Newton} X-ray spectroscopy of \citet{peterson03}, with 
 1$\sigma$ error bars. An empirical model is fit to the data, and plotted 
in red. The standard cooling flow model is plotted in green. Clearly, 
the uninhibited cooling flow model severely 
overpredicts cooling line fluxes stemming from the soft X-ray gas. 
This figure is reproduced from Fig.~4, panel 2 in \citet{peterson03}, with permission from Dr.~John Peterson. }
\label{fig:intro_petersonspec}
\end{figure}





While this  is strongly suggestive of an  intrinsic connection between
short  ICM cooling  times  and  the presence  of  these phenomena,  
debate continues as to whether these  active  gas
reservoirs    are   direct   condensates    from   a    cooling   flow (e.g., \citealt{odea08}, and references therein), or are  instead deposited through wet mergers
or    gas-rich     tidal    stripping    from     nearby    companions
(e.g., \citealt{sparks89,holtzman96}, and references therein).
%\citep{heckman81,hu85,johnstone87,romanishin87,romanishin88,heckman89,mcnamara89,mcnamara92,mcnamara93,crawford92,crawford93,allen95,mcnamara97}.
If it were the latter case,  a preferentially high gas rich merger rate in cool
cores would be needed to  
reconcile observations that CC BCGs are more than three times as  likely as non-CC BCGs  to host emission  line nebulae ($\sim
45$\%             vs.~$\sim            10$\%,            respectively,
\citealt{crawford99,best06,edwards07}) and central radio sources 
($\sim 70$\% vs.~$\sim  20$\%, respectively,
\citealt{burns90}), though  it is not  obvious why this should  be the
case.





On the  other hand, it is now clear that the emission line nebulae 
are not simply the $10^4$ K phase of a cooling flow. 
While the bulk of the ``baryon budget'' for  these phenomena
might be accounted for by condensation from the ambient hot reservoir, 
the cooling flow model in isolation 
seriously underpredicts the observed H$\alpha$ luminosities by orders of magnitude
in a pure radiative cooling scenario \citep[e.g.,][]{heckman89,voit97}. 
The nebulae are also
characterized  by  low ionization  forbidden  line  flux that  greatly
exceeds the flux emerging from recombination lines, requiring 
models to impart a great deal of heating per ionization event \citep{donahue91,voit97}. 
The low levels (a few \Msol\ yr\mone) of star formation observed in half of all CC BCGs
can often account for the ionizing photons required to power the observed H$\alpha$ luminosity \citep[e.g.,][]{bildfell08,odea08,odea10,mcdonald10,mcdonald11}, but another heating mechanism is required to account for the inverted forbidden-to-Balmer and other diagnostic line ratios \citep[e.g.,][]{voit97}. Collisional heating by suprathermal electrons (cosmic rays) has gained 
favor in recent years (\citealt{ferland09,donahue11,fabian11}; Mittal et al.~2011, in press). 



\subsection{Heating cluster cores with ``effervescent'' AGN feedback and 
conduction}


While the cooling flow model is inconsistent with observations, there  is no doubt that  the ICM in CC  clusters is losing
energy at  a rate proportional to  its X-ray luminosity,  which can be
extreme in some cases  ($\lae 10^{45}$ ergs sec\mone).  A compensatory
heating  (or  reheating)  mechanism  is therefore  needed  to  balance
cooling in the majority of cases,  thereby accounting  for the  dearth of  predicted warm and cold mass
sinks  associated with  the  cooling flow model.   CC  clusters account  for
$>50\%$  of  the  X-ray  luminous  cluster  population  below  $z<0.4$
\citep[e.g.,][]{mcnamara07},  so  they  must  be  heated  at  an  average  rate
approximately of  order their X-ray  luminosity (heating rates  far in
excess  of  this  would  push  cooling times  past  the  Hubble  time,
destroying the  CC phase).  Such  a scenario might naturally  arise if
the   heating  and   cooling   rates  were   tightly   coupled  by   a
self-regulating feedback mechanism (e.g., \citealt{begelman01,ruszkowski02,churazov02,reynolds02,birzan04}).


\begin{figure}
\begin{center}
\includegraphics[scale=0.48]{perseus_xray.eps}
\includegraphics[scale=0.21]{perseus_composite.eps}
\end{center}
\caption[AGN feedback in the Perseus cluster]{{\it Chandra X-ray Observatory} X-ray image ({\it top}) and {\it Hubble Space Telescope} (optical, red filaments), {\it Chandra X-ray Observatory} (X-ray, in blue), and Very Large Array (Radio, in pink) composite image ({\it bottom}) of the central brightest cluster galaxy in the massive Perseus cluster. This is a canonical example of AGN feedback associated with a radio source subsonically excavating kiloparsec-scale buoyant cavities into the surrounding hot intracluster gas. 
AGN feedback is currently the favored mechanism by which catastrophic radiative cooling of the intracluster gas is balanced by heating associated with the $p~dV$ work done during 
cavity excavation by the radio source. This energy can exceed 10$^{61}$ ergs, enough to balance radiative losses. Spatial distribution of this energy is one of many important problems challenging the model.}
\label{fig:intro_fabian}
\end{figure}






Conductive thermal interfaces that  impart radiative inefficiency is a
natural       and       energetically       feasible       explanation
\citep{bertschinger86,bregman88,sparks89, sparks92},  though debate is
ongoing as to whether or not conduction-balanced radiative cooling can
establish  stable   temperature  and  density   gradients  that  match
observations for necessarily long  timescales ($\gg1$ Gyr), due to the
strong     temperature-dependence     of     conduction     efficiency
\citep{spitzer62,cowie77,fabian94,soker03,voit05}.   Conversely,  pure
radiative cooling  does not  establish the observed  gradients either,
unless conduction efficiency is suppressed by a (possibly) unrealistic
two                 orders                 of                magnitude
\citep{bregman88,malyshkin01,narayan01}.  Conduction  therefore cannot
be dismissed off-hand,  and may play a particularly  important role in
the       outer       regions       of       the       cool       core
\citep{sparks89,sparks92,ruszkowski02,brighenti03,sparks09}.




  
\citet{rosner89} and  \citet{baum91} were  among the first  to suggest
that  the dissipation  of active  galactic nucleus  (AGN)  power could
fully  replenish  ICM  radiative  losses in  regions  near  radio-loud
BCGs. This  AGN-driven feedback mechanism, in  concert with conduction
and  heating  by  cosmic  rays,  has since  become  the  most  favored
candidate  model  by  which  cool  cores are  heated.  The  radio-mode
feedback  model is  largely  motivated by  observations of  kiloparsec
scale  X-ray cavities  in strong  spatial anti-correlation  with radio
emission          stemming          from         AGN          outflows
(e.g., see Fig.~\ref{fig:intro_fabian} and \citealt{boehringer93,fabian00,fabian06,churazov01,mcnamara00,mcnamara01,blanton01,nulsen05,forman05,forman07,birzan04}).




In  the  general  radio-mode  AGN  feedback  model,  the  cooling  ICM
collapses into the BCG, reaching  the center and triggering black hole
(BH)  activity.   The associated  outflowing  plasma can  subsonically
excavate  cavities  in  the  thermal  gas, driving  sound  waves  that
eventually dissipate  into heat \citep{fabian03}.   These cavities are
effectively lower  density ``bubbles''  which buoyantly rise  amid the
ICM,  entraining colder gas  phases and  magnetic fields  (required to
keep them long-lived, e.g., \citealt{robinson04,dursi08}), lowering the total mass inflow rate in the cool
core  and thermalizing  cavity enthalpy  as the  ICM refills  its wake
\citep{begelman01,ruszkowski02,churazov02,reynolds02,birzan04}.    Over
the  cluster  lifetime, enthalpy  dissipation  associated with  cavity
inflation  can  range  from $\sim10^{55}-10^{61}$  ergs,  theoretically
enough  to counter  radiative  losses  in the  ICM  on average  (e.g.,
\citealt{birzan04,rafferty06}),  though spatial  distribution  of this
energy is one  of several important problems that  challenge the model
(see e.g., \citealt{mcnamara07}, for a review).  Moreover, AGN heating
is almost certainly episodic at a  rate coupled to the AGN duty cycle\footnote{We define the ``AGN duty cycle'' to be the fraction of total time that the AGN is ``on''.} ($10^7-10^8$ yr).
Not only is this naturally  expected, it is effectively {\it required}
in  order to  reconcile the  model  with observations  of (1)  entropy
gradients   that  increase  monotonically   with  CC   cluster  radius
\citep{kaiser03},  (2)  the  $\sim  30$\%  of CC  clusters  that  lack
detectable X-ray cavities \citep{dunn06}  and (3) some ``ghost'' X-ray
cavities that are apparently devoid of cospatial radio emission (e.g.,
Abell 2597, \citealt{mcnamara01}).
Regardless, AGN  feedback is an attractive heating  candidate, and recent years
have seen feedback  emerge as a critical  component in models of galaxy
evolution (e.g.,   \citealt{springel05}).   The   addition   of
feedback-driven ``anti-hierarchical''  quenching  of   star  formation  may
elegantly regulate  the possibly coupled  co-evolution  of black  holes  and the
stellar component           of          their           host          galaxies
\citep{magorrian98,ferrarese00,gebhardt00}, in  addition to truncating the
bright  end of the  galaxy luminosity function and  explaining the bimodality
of     galaxies      in      color-magnitude     space
\citep[e.g.,][]{silk98,scannapieco05,bower06,croton06,sijacki07}.






While only 20\% of non-CC BCGs are radio loud, 70\% of CC BCGs harbor active
central radio sources, \citep{burns90,ball93,mittal09,sun09}, the  power of
which  appears to correlate  with the X-ray  luminosity from  within the
cooling radius (e.g., \citealt{birzan04,mittal09}).  Considering this, an
elegant scenario emerges wherein condensation  from  a  rapidly  cooling
ambient atmosphere may  directly provide substantial components of the fuel
reservoir for multiple  episodes of AGN activity, in addition to  the  active
central  emission  line nebulae  and  star  formation residing  amid pools of
cold gas. In turn, energy input from the AGN regulates the pathway of entropy loss
by heating the ambient gas such that the heating rate tightly regulates the
cooling rate (and vice-versa) in a feedback loop.  


\subsection{Some important outstanding issues}


Nevertheless, fundamental open questions remain. The detailed physics
of feedback are still not understood. It is still not known whether 
the spectacularly mysterious 
emission line filament complexes are ``dragged upward'' by radio sources 
or are associated with infalling cold gas \citep[e.g.,][]{mcdonald10,mcdonald11,fabian11}.  
As discussed, additional heating mechanisms such
as photoionization from young stars and possibly collisional heating from energetic
particles is required to regulate the observed temperature and ionization state
of the warm and cold gas phases. 
AGN feedback is generally invoked as a mechanism by which star formation is {\it quenched}, 
and yet there is some evidence that, at least on small size scales, the propagation 
of a radio source through a dense molecular medium can {\it trigger} localized sites 
of star formation \citep[e.g.,][]{vanbreugel04,odea04}. Finally, 
if a substantial component of the ISM in CC BCGs has indeed condensed from a cooling flow, it is  
clear that AGN feedback cannot 
establish an impassable ``entropy floor'', and 
completely  inhibit cooling in
all  CC BCGs  at all  times. Some cooling must manage to persist, 
either during quiescent periods of AGN inactivity, or at constant low levels.   
Discriminating between the many scenarios requires a better understanding of 
all temperature phases of the ICM, the transport processes between 
these phases, and their associated mass and energy budgets. Our ability to do so 
is critically dependent upon multiwavelength data that sample these various discrete 
temperature phases. 



This is a short and highly incomplete contextual summary of the issues 
this thesis seeks to address, at least in part. 
As a necessary primer for better understanding of the results we will present, 
the section below provides an abridged review of some of the finer historical and technical 
details pertaining to galaxy clusters and cool cores. 




\begin{figure}
\plotone{hydra.eps}
\caption[X-ray supercavities in the Hydra A galaxy cluster]{The 100 kpc-scale supercavity system in Hydra A. \citet{wise07}
estimate that the radio source has returned upwards of $10^{61}$ erg to regions
of the cluster that extend even beyond the cooling radius. Note the strong
spatial anti-correlation of the radio source (in pink) with the X-ray cavities
(in blue). Such correspondence is typical of cool core clusters with cavities
and extended radio sources, presenting strong circumstantial evidence in
support of the AGN feedback model. See \citet{birzan04} for a comprehensive
census of the currently known X-ray cavity systems in groups and clusters of
galaxies. (Credit: X-ray: NASA/CXC/U.Waterloo/C.Kirkpatrick et al.; Radio:
NSF/NRAO/VLA; Optical: Canada-France-Hawaii-Telescope/DSS).}
\end{figure}



\clearpage




\section{A Review of Galaxy Clusters, the Intracluster Medium, and AGN Feedback}




The previous section summarized the contextual background directly relevant to
this thesis, and took many opportunities to sacrifice detail for the sake of
brevity.  Our goal was to provide an introduction to the various issues
addressed in this thesis, without distracting the reader with too much
secondary or background detail.  Of course, these ``details'' are critical to
understanding of the topics at hand, so we attempt to summarize
them here, at least in part. This should in no way be mistaken for a
comprehensive review; rather, it is more a ``brief tour'' of some of the major
aspects that define our current understanding of galaxy clusters, particularly
with regards to the cooling and heating models that attempt to describe the
observed properties of the intracluster medium.  While we devote more attention
to the AGN feedback model than we do alternative heating scenarios (like
thermal conduction), this should not be construed as advocacy for one model
over another, as that is not the point of this thesis.  In each section, we
will point the reader to more comprehensive reviews relevant to the subject
being discussed.







\subsection{Galaxy clusters in a cosmological context}




Galaxy clusters are the largest gravitationally collapsed structures in the
Universe\footnote{Throughout this thesis we adopt 
the concordance ($\Lambda$CDM) cosmological model, with 
$H_0   = 71$  km
s$^{-1}$  Mpc$^{-1}$,   $\Omega_M  =   0.27$,  and $\Omega_{\Lambda} = 0.73$.
The Hubble constant $H_0$ scales the relationship between recessional velocity
$v_r$ and distance $D$ in the expansion of the Universe, such that $v_r=H_0 D$. $\Omega_M$ 
is the ratio of the mass density and  $\Omega_{\Lambda}$
is the ratio of the energy density due to Einstein's cosmological constant to the 
critical density of the Universe. Parameterized in this fashion, the Universe is 13.7 Gyr old, and was 5.9 Gyr old at a redshift of $z=1$.}, with masses of order  10$^{15}$ \Msol\ and volumes that span hundreds
of cubic Mpc. The tens to thousands of galaxies inhabiting this volume encode a history 
of hierarchical structure assembly in the Universe. 
At the earliest epochs, small scale,
large amplitude baryonic density perturbations exceeding the mean density of the rapidly
expanding Universe decouple from the Hubble flow and gravitationally collapse.
These sub-stellar clumps coalesce, triggering hierarchical structure growth
which proceeds non-linearly by the accretion of ever larger and more complex
arrangements of baryonic matter, including stars, stellar clusters, and eventually galaxies.  
Galaxy clusters are the final manifestation
of this 13 billion year-long process, forming at late times (redshifts $z <
2$) in massive dark matter halos. 
Just as for the baryonic component of the Universe, dark matter structure also grows hierarchically, 
and the largest halos (hosting the richest galaxy clusters) are found at the intersections 
of large scale dark matter filaments 
(see, e.g.,
Fig.~\ref{fig:intro_springel} and \citealt{springel05}).



Many of the important properties of galaxy clusters are expressed in terms of a
cluster's ``virial radius'', which can be understood in the context of a simple
``violent relaxation'' model \citep[e.g.,][]{lyndenbell67}. During cluster
assembly, gravitational interaction between infalling baryonic clumps produces
a time-variable gravitational potential that yields randomized particle
velocities following a roughly Maxwellian distribution. The result is a state of 
``virial equilibrium'', in which the total kinetic energy $E_K$ and gravitational 
potential energy $E_G$ are related by 
\begin{equation}
2 E_K + E_G = 4\pi P_\mathrm{b} r_\mathrm{b}^3,
\end{equation}
where $P_\mathrm{b}$ is the effective pressure at $r_\mathrm{b}$, the outer 
boundary of the relaxed system. If $P_\mathrm{b}=0$, we recover the usual 
form of the virial theorem, e.g., $2 E_K + E_G = 0$.   
The outer boundary ($r_\mathrm{b}$) of a cluster is not especially well-defined in practice,
and is therefore not measurable in the strictest sense. 
It can, however, be estimated with a spherical ``top-hat'' model, wherein  
a uniform, constant density sphere is assumed to cause the perturbation 
leading to cluster collapse. In this case, 
the virial theorem posits that a collapsed cluster's bounding radius should be 
of order 0.5 times the turnaround radius $r_\mathrm{turn}$. 
If all mass in the relaxed cluster resides within  $r_\mathrm{turn} / 2$, 
then the mass density within this sphere is $6 M / \pi r_\mathrm{turn}^3$, 
which for an Einstein-deSitter (flat, matter-dominated) cosmology 
is $\sim 178$ times the critical density of the Universe, defined as
\begin{equation}
\rho_\mathrm{crit} \equiv \frac{3 H^2 }{8\pi G}.  
\end{equation}
This is the density threshold above which the expansion of the Universe 
would stall, leading to recollapse. 
Here,  
$G$ is Newton's gravitational constant, and $H$ accounts for the redshift dependence of the Hubble constant $H_0$, 
\begin{equation}
\frac{H}{H_0} = \sqrt{\Omega_M \left(1+z\right)^3 + 1 - \Omega_M},
\label{equation:hubblescale}
\end{equation}
where $\Omega_M$ is the ratio of the mass density of the Universe to its critical density. The two quantities are equal (e.g., $\Omega_M=1 $) in an Einstein-deSitter Universe. In this case, the radius enclosing $178 \rho_\mathrm{crit}$ would 
be the virial radius $r_\mathrm{vir}$. 
Sometimes, alternative definitions are used, such as 
$r_\mathrm{vir}=r_{200}$. In this nomenclature, 
$r_N$ is defined to be the radius at which the average cluster density is $N \times \rho_\mathrm{crit}$, so
\begin{equation}
r_{200} \equiv 200 \times \frac{3H^2}{8\pi G}.
\end{equation}
The virial radius has also been approximated as the ``scale radius'' 
\begin{equation}
r_{180M} = 180 \times \Omega_M \left(z \right) \rho_\mathrm{crit}. 
\end{equation}
If $\Omega_M \approx 1$, these alternative definitions are 
very close to the $r_\mathrm{vir} = 178 \rho_\mathrm{crit}$ derived in the Einstein-deSitter case, so $r_{200} \approx r_{180M} \approx r_\mathrm{vir}$. However, $\Omega_M \approx 0.3$ in
the currently favored $\Lambda$CDM cosmology, which gives $r_\mathrm{vir} \approx 100 \rho_\mathrm{crit}$. Nevertheless, $r_\mathrm{200}$ and $r_{180M}$ are still 
used throughout the literature, for the sake of consistency.  
Regardless of the definition 
used, typical virial radii for rich clusters are on the order of $1-3$ Mpc.




As they are the largest of all structures whose mass can be reliably estimated, 
galaxy clusters are among the best laboratories in which cosmological parameters and
galaxy formation models can be constrained by observables.  
This thesis is broadly related to the latter. For a more cosmologically-oriented 
review of galaxy clusters, see \citet{rosati02} and \citet{voit05}. For a review of the critically important Sunyaev-Zeldovich effect 
\citep{sunyaev70,sunyaev72} as it pertains to galaxy clusters and cosmological probes, see e.g., \citet{carlstrom02}. 





\subsection{Properties of the intracluster medium}


Galaxy clusters were first associated with extended, luminous X-ray sources in the late 1960s and 1970s
after the launch of the Uhuru, Ariel 5, and OSO 8 X-ray satellites \citep{giacconi71,gursky71,forman72,mitchell76,gursky77,serlemitsos77}. 
In observing the Perseus, Coma, and Virgo clusters, \citet{mitchell76} and \citet{serlemitsos77} were the first to confirm a thermal origin for the X-ray emission
in detecting collisionally excited Fe-K features above 6 keV. 


At the same time, our understanding of hierarchical structure formation was maturing (e.g., \citealt{silk68,gunn72,silk77,rees77}). This motivated
the interpretation that 
the ICM itself, which is largely composed of ionized hydrogen and helium, is a
relic from the gravitational collapse at early times of ``proto-baryons'' that
do not participate in star and galaxy formation (\citealt{gunn72} were among the first to suggest this). We now know that $\sim85\%$ of all baryons in galaxy clusters reside in the ICM, 
though not {\it all} of them can be primordial. A significant fraction of the ICM
needs to have been processed in one or more generations of star formation in order to account for 
the presence of  heavier elements in the ICM at abundances $\sim0.4$ solar (e.g., \citealt{arnaud92}). This can be explained by invoking the 
return of higher metallicity gas from supernovae, as well as ram pressure and tidal stripping 
from galaxies as they fall into the cluster.

For comprehensive reviews of the ICM, see \citet{sarazin86}, \citet{mushotzky04}, and \citet{arnaud05}. 
What follows is largely an abridged version 
of topics covered at length in these reviews. 



\subsubsection{{\bf Radial density and mass profiles}}

Models of the ICM treat it as an optically thin coronal plasma\footnote{The term ``coronal plasma'' refers to the ``coronal approximation'' described by e.g., \citet{mewe99}, wherein 
electron-driven collisional ionization rates are exactly balanced by recombination rates. 
See \citet{sarazin86} and \citet{peterson06} for more details as they pertain to galaxy clusters.} in ionization equilibrium. 
Observed (measured) particle densities $n$ in the ICM are of order $10^{-4}-10^{-2}$
cm$^{-3}$, and increase from the outer halo inward to the cluster core. 
The radial density profile depends on many subtle physical processes (the most important of which 
we will summarize later), but will generally follow hydrostatic equilibrium to maintain pressure support. 
Assuming spherical symmetry, the hydrostatic gas pressure profile will rise with decreasing radius as 
\begin{equation}
\frac{dp}{dr} = \frac{d\left(nkT\right)}{dr} = - \rho g  = -n \mu m_p \frac{G M\left(<r\right)}{r^2},
\end{equation} 
where $k$ is the Boltzmann constant, $T$ is the gas temperature, $\rho=n \mu m_p  $ is the mass density, $g$ is the local gravitational 
acceleration, $M\left(<r\right)$ is the mass enclosed within radius $r$, $\mu m_p$ is the mean mass per particle, 
and $m_p$ is the proton mass. The hydrostatic approximation allows for rough estimates of cluster mass profiles, e.g., 
\begin{equation}
M\left(<r\right) = -\frac{kTr}{G \mu m_p} \left( \frac{d \log n_e}{d \log r} + \frac{d \log T}{d \log r} \right).
\end{equation}
Observations can be used to parameterize the density (and therefore temperature) profiles in the above equations. 
A common method is to fit an isothermal ``beta'' ($\beta$) model to the observed X-ray surface brightness profile, 
as first introduced by \citet{cavaliere76,branduardi81,forman82}. 
In the beta model, the X-ray surface brightness $\Sigma_\mathrm{X}$ at projected radius
$r$ varies as 
\begin{equation} 
\Sigma_\mathrm{X}\left(r\right) =
\Sigma_{\mathrm{X}}\left(0\right) \left[ 1 + \left(\frac{r}{r_0}\right)^2\right]^{\left(-3
\beta + 1/2\right)}. 
\end{equation} 
Here, $\Sigma_{\mathrm{X}}\left(0\right)$ is the central
X-ray surface brightness and $r_0$ is the core radius.  The model assumes that
the galaxies, ambient hot gas, and underlying dark matter halo (assumed to follow a King profile) are hydrostatic and isothermal. 
$\beta$ is defined as the ratio of galaxy-to-gas velocity dispersions (effectively  the energy per unit mass in galaxies divided by that in gas)
, 
\begin{equation} \beta \equiv \frac{\mu m_p \sigma^2}{kT}, \end{equation}
Here, $\sigma$ is the 1-D velocity dispersion of the galaxies along the line of sight.  
Typically, the ICM in galaxy clusters is well-fit by models with
$\beta\approx0.6$ \citep{sarazin86}. 
The beta model form for X-ray emissivity can be converted to electron density using 
\begin{equation}
n_e\left(r\right) =
n_e\left(0\right) \left[ 1 + \left(\frac{r}{r_0}\right)^2\right]^{\left(-3\beta / 2\right)}. 
\end{equation}





\subsubsection{{\bf Temperature}}


At early epochs, adiabatic compression and supersonic accretion shocks are thought to heat the ICM to its 
observed temperature \citep{kaiser86}, which ranges from $10^7 \lae T \lae 10^8$ K ($1 \lae kT \lae 10$ keV). This is  
roughly the virial temperature of the underlying gravitational potential, so 
\begin{equation}
kT \approx \frac{GM m_p}{r_\mathrm{vir}} \simeq \mu m_p \sigma^2 \simeq 6 \times \left( \frac{\sigma^2}{10^3~\mathrm{km~s}^{-1}}  \right)~\mathrm{keV} .  
\label{equation:temp}
\end{equation} 
 


Combining equations \ref{equation:temp} and \ref{equation:hubblescale} and applying simple manipulations 
(the details for which can be found in e.g., \citealt{rosati02}), we can relate the ICM temperature 
to the virial mass of the cluster $M_v$
(e.g., the mass contained within $r_v$) by 
\begin{equation}
kT = 1.38 \left( \frac{M_\mathrm{vir} h}{10^{15} M_\odot}    \right)^{2/3} \left[\Omega_m \Delta_\mathrm{vir}\left(z\right)\right]^{1/3}  \left( 1 + z \right) ~ \mathrm{keV}. 
\label{equation:tvir}
\end{equation}
Here, $ h$ is the dimensionless Hubble parameter\footnote{The dimensionless Hubble parameter $h$ is related to the Hubble constant $H_0$ by $ h = H_0 / ( 100 $ km s$^{-1}$ Mpc$^{-1})$.} allowing for simple conversion between cosmologies, 
and $\Delta_\mathrm{vir} \left(z\right)$ is the average cluster density within the virial radius divided by the mean cosmic density 
at redshift $z$. 
If $\Delta_\mathrm{vir} \left(z\right)$ is constant (as it is for e.g., an Einstein-de-Sitter cosmology), then the 
net implication is that there is a fundamental scaling relation 
\begin{equation}
T \propto M^{2/3} \left( 1 + z \right).
\label{equation:tmass}
\end{equation}
This is one of the several examples in which quantities measured from X-ray observations (like density and temperature)
can be used to infer cluster mass, and therefore constrain cosmological models which make independent cluster 
mass predictions. 


\subsubsection{{\bf Spectrum}}


 
The dominant coolant for ionized hydrogen at $>10^7$ K is thermal bremsstrahlung (free-free) emission 
driven by Coulomb interactions between electrons and ions. This powers an extremely bright X-ray continuum
of luminosities $L_X \sim 10^{43}-10^{45}$ erg sec$^{-1}$, 
with the total power radiated per unit volume $V$ given roughly by 
\begin{equation}
\frac{dL}{dV} \approx 10^{-27} n_e n_H T^{1/2} \mathrm{erg~sec^{-1}~cm^{-3}},
\label{equation:brem}
\end{equation}
where $n_e$ and $n_H$ are the electron and hydrogen densities in cm$^{-3}$, respectively.
In a fully ionized gas with hydrogen and helium mass fractions of $X=0.7$ and $Y=0.28$ (respectively), 
the electron and hydrogen densities are approximately equal ($n_e\approx 1.18 n_H$). 
The energy loss rate by bremsstrahlung emission therefore scales with the square of the gas density 
(this has very important consequences, which we will discuss later).
Below about $3\times10^7$ K, cooling by iron, oxygen, and silicon recombination lines becomes increasingly important 
and significantly alters the shape of the X-ray spectrum. Prominent X-ray spectral features include the Fe K and L lines at $\sim 6$ and $\sim 1$ keV, 
respectively. These are used in X-ray spectral fitting (which plays an important role in our Chapter 2) to ascertain elemental abundances in the ICM 
(which, as mentioned previously, are usually $\sim 0.3-0.4$ solar). 





\subsubsection{{\bf Entropy}}



One of the most important physical quantities dictating the structure and density of the ICM 
is entropy, unique in its ability to encode and preserve the thermodynamic history of the gas. Considered alone, temperature and density fail in this regard, 
as the temperature largely reflects the underlying gravitational potential well, and 
the density reflects the degree to which gas is compressed within the well.
Entropy, however, only changes via gains or losses of heat energy, and breaks the degeneracy with the underlying potential because 
at constant pressure, the density of the gas is determined by 
its specific entropy. Quantitatively, the entropy $S$ can be defined by rewriting the expression 
for the adiabatic index ($K\propto p\rho^{-5/3}$) in 
terms of the observables $kT$ and $n_e$,  
\begin{equation}
S=k T n_e^{-2/3}. 
\end{equation}
One may also write
\begin{equation}
K \equiv \frac{k T}{\mu m_p \rho_g^{2/3}},
\end{equation}
where $K$ is merely the proportionality term in the equation of state for an 
adiabatic monatomic gas, $p = K \rho_g^{5/3}$. $K$ is also directly related to the standard thermodynamic definition of entropy per particle, 
namely $s= k \ln K^{3/2} + s_0$. 
One may convert from one definition to the other 
by 
\begin{equation}
S = k T n_e^{-2/3} = 960~\mathrm{keV~cm}^2 \left( \frac{K}{10^{34}~\mathrm{erg~cm}^2 g^{-5/3}} \right). 
\end{equation}
Because high 
entropy gas ``floats'' and low entropy gas ``sinks'' \citep{voit05}, 
the intracluster gas will convect until it establishes 
an entropy gradient that corresponds to the underlying gravitational potential. 
In other words, convection will attempt to establish a match between the    
isentropic surfaces of the ICM and the equipotential surfaces of the DM halo, and will reach 
convective equilibrium when $dK/dr \ge 0$ everywhere (e.g., \citealt{voit05}, and references therein).
There are many important works which focus specifically on intracluster entropy and its many consequences,  see 
e.g., \citet{lloyd00,voit02,piffaretti05,pratt06,donahue06,cavagnolo09}, and references therein.


%\footnote{There are widespread misconceptions of entropy, most of which stem from poorly constructed analogies 
%that attempt to explain the second law of thermodynamics. This may lead some to question 
%the many instances throughout this thesis wherein we refer to ``entropy loss'' as it pertains to cooling flows 
%and star formation. In fact, there is nothing wrong with the notion of localized entropy loss at the expense of a global 
%entropy gain. For proof, one need not look further than the nearest refrigerator. }.

\subsubsection{{\bf Scaling relations}}



If the ICM is entirely characterized by gravitational processes and bremsstrahlung emission, 
and the cosmological model is scale-agnostic, then all clusters of varying masses can be 
assumed to be scaled versions of the same ``unit phenomenon''.
One may then apply consequences arising from self-similarity, and 
construct simple scaling relations following arguments similar to those yielding equation \ref{equation:tvir}. 
For example, 
as shown in equation \ref{equation:brem}, the X-ray luminosity $L_X$ arising from pure bremsstrahlung emissivity should scale with 
cluster mass and temperature as 
\begin{equation}
L_X \propto M \rho_g T^{1/2} \propto T^2 \left(1+z\right)^{3/2} \propto M^{4/3} \left(1+z\right)^{7/2}.
\end{equation}
Similarly, entropy should scale with temperature as 
\begin{equation}
S \propto T\left(1+z\right)^{-2},
\end{equation}
and so on. 


Decades of observations have shown these self-similar scaling relations to be incorrect. 
For example, the actual observed luminosity-temperature relation for the $3-10$ keV range is of order 
$L_X \propto T^{3}$ rather than $\propto T^2$  \citep{markevitch98,arnaud99}, and is even steeper for $<1$ keV groups of galaxies.
 The breaking of self-similarity in scaling relations 
could possibly be explained by {\it pre}-heating of the gas prior to virialization at early epochs, which would provide an 
entropy excess needed to explain this departure.  
Early epochs of AGN activity and metal-enriching supernovae are proposed pre-heating candidates (see \citealt{rosati02,voit05}, and references therein, for far more 
detail than we present here).  


\subsubsection{{\bf Magnetic Fields and Thermal Conduction in the ICM}}

\label{section:magneticfields}

Faraday rotation measurements of radio galaxies embedded within clusters confirm that the ICM 
is permeated by magnetic fields of a few to tens of $\mu$G in strength. 
The strongest fields are found in cool core clusters \citep{clarke01}, and are thought to be 
(1) primordial, initially weak fields amplified over time by intracluster turbulence, convection, and 
adiabatic compression, and / or (2) injected by radio galaxies. Even for the strongest fields, 
the gas pressure greatly dominates over the magnetic pressure, e.g., 
\begin{equation}
nkT \gg \frac{B^2}{8\pi},
\end{equation}
but one must still not underestimate the importance of magnetic fields in regulating the global physics of the ICM (see, e.g., \citealt{soker10}, and references therein). For example, an
electron in the ICM will strongly couple to the magnetic field because its mean free path $l_e$ is far larger 
than its Larmor radius $\rho_e$, 
\begin{equation}
\frac{l_e}{\rho_e} \sim 10^{14} \left( \frac{B}{1~\mu\mathrm{G}} \right) \left(\frac{n_e}{10^{-2}~\mathrm{cm}^{-3}} \right) \left( \frac{T}{3~\mathrm{keV}} \right).
\end{equation}
This will have 
significant impacts on transport coefficients like viscosity, and give rise to anisotropic 
heat conduction whose efficiency is almost entirely regulated by the magnetic field. 
The degree to which this efficiency may be suppressed (e.g., \citealt{tribble89}) is an area of rigorous debate
(to put it mildly). We will briefly touch upon this later in discussing thermal conduction as it applies to inhibiting cooling flows. 
We hesitate to so briefly summarize such an important area of study, but it is at least worth noting that magnetic 
fields have been invoked to solve problems related to (1) the perplexing longevity of buoyant X-ray cavities, which should 
disrupt on the order of a single sound crossing time (e.g., \citealt{dursi08}, see Section~\ref{section:bubbledisrupt}), as well as (2) the mysterious filamentary morphologies of emission line nebulae in CC BCGs (e.g., \citealt{fabian08}) and (3) the observed presence of large radio relics in the ICM (e.g., \citealt{markevitch05}). 






\subsection{The classical cooling flow model}
\label{section:coolingflows}



If the hot ICM is optically thin to its own radiation, it will lose energy 
at a rate directly proportional to its X-ray luminosity. If this energy loss 
is not balanced by any other non-gravitational forms of heating, the gas will 
radiate away all of its energy on a timescale set by the gas enthalpy divided by 
the energy lost per unit volume. The timescale over which this should occur, $t_\mathrm{cool}$, can be expressed in terms of the gas pressure $p$, 
\begin{equation}
t_\mathrm{cool} = \frac{p}{\left[\left(\gamma -1   \right)  n_e n_H \Lambda\left(T\right) \right]},
\label{eqn:tcool}
\end{equation}
where $\gamma$ is the ratio of specific heats and $\Lambda\left(T\right)$ is the cooling function,
which integrates all emission processes in the plasma (dominated by bremsstrahlung in the keV gas), 
weighted by photon energy. See, for example, the ICM cooling functions of \citet{sutherland93,ruszkowski02}. 


Cool core clusters of galaxies are categorized as such because the radiative lifetime ($t_\mathrm{cool}$) of their ICM 
within the ``cooling radius'' is shorter than the Hubble time, e.g., $t_\mathrm{cool} < H_0^{-1}\sim13.7$ Gyr.
The X-ray luminosity within the $\sim 100$ kpc-scale cooling radius $r_\mathrm{cool}$ can reach $10^{45}$ erg s$^{-1}$ 
in the most extreme cases, for which   $t_\mathrm{cool} \approx 3\times10^8$ yr (which is only 1/30th the Hubble time). 
The pressure at the cooling radius $r_\mathrm{cool}$, beyond which cooling can be considered unimportant, 
is set to first order by the weight of the overlying gas. 
Energy loss by gas within the cooling radius is accompanied by a localized drop in entropy, 
but it still must maintain pressure support for these outer layers. This can only happen if its density increases, 
which, to first order, is only possible if it compresses and flows inward. As is obvious from Equation \ref{eqn:tcool}, 
the cooling time is inversely proportional to the gas density. This leads to a ``runaway'' effect, wherein the cooling {\it rate} 
of gas within $t_\mathrm{cool}$ increases catastrophically. 
Cooling times are nonetheless much longer than free-fall times at common radii, 
and gas ``lost'' to the cooling flow is constantly replenished by the ambient hot reservoir. 
Thus, the cooling flow is subsonic, quasi-hydrostatic, pressure-driven, and long-lived (e.g., \citealt{lea73,cowie77,fabian77};
see reviews by \citealt{fabian94,peterson06}).  


As the name implies, entropy loss within the cooling radius is associated with a loss of heat, 
but surprisingly, this does not necessarily mean that the {\it temperature} of the gas should decrease 
inwards as a direct result of the cooling flow. Rather, the gas will cool only 
if its initial temperature is greater than the local virial temperature of the underlying gravitational potential. 
When the two temperatures are equal, gravity will ``take over'', and adiabatic compression associated with the release of gravitational 
energy will act as a thermostat for the gas, keeping it at roughly the local virial temperature, i.e.,
\begin{equation}
\frac{kT}{\mu m_H} \approx \frac{G M\left(<r \right)}{r}.
\end{equation}
This means that, counter-intuitively, the ``cooling flow'' is effectively isothermal. 
The observed (and predicted) temperature of the gas {\it does} drop inwards toward the BCG, 
but this is less a direct consequence of the cooling flow, and is instead a 
reflection 
of the flattening gravitational potential as the BCG begins to dominate over the cluster potential at the innermost radii.
Stated another way, as the temperature of gas within the cooling flow matches the local virial temperature,
we observe cooler gas in the central regions because the virial temperature is lower in the BCG than it is in the ambient cluster.
Note that this simple observation is actually model-agnostic, meaning 
observed temperature profiles are effectively unrelated to the cooling flow model, regardless of whether or not it is correct in principle. 
The term ``cooling flow'' is therefore something of a misnomer, and is better thought of as a runaway cascade of 
localized {\it entropy} loss governed by 
\begin{equation}
\frac{ds}{dt} = \frac{-n_e n_H \Lambda\left(T\right)}{\rho T}.
\end{equation} 


To first order, the X-ray luminosity associated with a perpetual, single phase, isobaric cooling flow can be calculated by 
finding the ratio of the enthalpy carried along with the flow to the gravitational potential energy dissipated within the flow, or  
\begin{equation}
L_X \simeq \frac{\dot{M} \left(5 kT\right)}{2\mu m_H} \simeq 1.3 \times 10^{44}  \left(  \frac{kT}{ 5~\mathrm{keV}}   \right) \left( \frac{\dot{M}}{100 ~M_\odot~\mathrm{yr}^{-1}}  \right) ~\mathrm{erg~s}^{-1},
\end{equation}
where  $\dot{M} = dM / dt$ is the X-ray derived mass deposition rate associated with the classical cooling flow. 
It is important to note that this predicted luminosity is roughly $10$\% of the gravitational potential energy release 
associated with a 0.02 \Msol\ yr$^{-1}$ accretion rate onto a supermassive ($10^8-10^9$ \Msol) black hole. We will discuss the important implications of this later.  



As mentioned previously, the cooling flow is always subsonic, though its Mach number will steadily increase inward until 
it reaches approximately the sound speed. Prior to this point, any growth of thermal instabilities 
associated with the flow are quickly ``smoothed out'' by buoyant motions \citep{balbus89}. 
The instabilities can persist for longer times as the flow becomes sonic, however, at which point 
Rayleigh-Taylor and shearing effects can disrupt the homogeneous flow into discrete, thermally unstable clouds
which decouple from the bulk flow. These instabilities, manifest as overdensities, then ``lag'' behind, persist, and grow. 
The result is expected to be a long-lived, inhomogeneous ``rain'' of cold gas clouds over a large central region within the CC BCG, which should 
then pool and collect over a small number of dynamical times.
Inhomogeneous cooling flow models such as these were historically more successful than the steady state, single-phase 
cooling flow models which overpredicted the central X-ray surface brightness \citep{nulsen86}.
As we summarize below, however, {\it major} problems remain.  



 



% ADD CITATIONS


%See \citet{fabian94,peterson06} for comprehensive reviews of cooling flows in clusters of galaxies. 


%\citealt{lea73,cowie77,fabian77};


\subsubsection{{\bf The ``failure'' of the simple cooling flow model}}


As discussed in section 1.1, the cooling flow model fails to match observations in three critical ways: 


\begin{enumerate}


\item It predicts that CC BCGs should possess massive repositories of cold gas of order $\sim10^{12}$ \Msol.
Sensitive searches for these cold mass sinks in the optical, IR, sub-mm, and radio returned empty-handed for many years. Since that time, 
some CC BCGs have been 
found to harbor substantial amounts ($10^9-10^{10}$ \Msol) of cold molecular hydrogen \citep{edge01,edge03,donahue00,salome06}. However, even in those cases when they are detected, 
these repositories 
are one to three orders of magnitude smaller than predictions. Moreover, cold molecular gas has so far only been 
detected in $15-30$\% of CC clusters (depending on the sample), though this could largely be due to the highly 
difficult nature of such observations.  



\item Such extreme cooling rates and massive predicted cold gas repositories should drive 
similarly extreme star formation rates up to $\lae10^3$ \Msol\ yr$^{-1}$,  two to three 
orders of magnitude in excess of results from three decades of observations.




\item There is {\it very little} high resolution X-ray spectroscopic evidence
that more than $\sim10$\% of the hot phase gas is cooling below $T_\mathrm{vir}
/ 3 \approx 1$ keV (e.g., \citealt{peterson03}). For example, the very
prominent and {\it expected} Fe {\sc xvii} line at 12 \AA\ is very weak or
completely absent in most CC clusters (except for A2597, the subject of the
next chapter).  Unlike {\it most} circumstances in astronomy (and life, as it
turns out), this is one instance in which ``absence of evidence'' (i.e., the lack or 
severe under-production of X-ray coolant lines) {\it really
is} ``evidence of absence''.  


\end{enumerate}







 
\begin{figure}
\begin{center}
\includegraphics[scale=0.38]{perseusgabeny.eps}\\
\includegraphics[scale=0.38]{perseusfabian.eps}
\end{center}
\caption[Another view of the Perseus cluster ({\it Chandra} X-ray, ground-based Optical)]{Another view of the Persesus cluster (and NGC 1275, center left) in optical ({\it top}) and X-ray ({\it bottom}).
The two fields of view are identical. The top optical image is a ground-based RGB+H$\alpha$ composite
by R.~Jay GaBany, reproduced here with his permission. The X-ray image is a 1.4 Msec {\it Chandra} exposure kindly provided by A.~Fabian and J.~Sanders, originally published in \citet{fabian11}.   }
\label{fig:intro_gabany}
\end{figure}




These ``three nails in the coffin'' gave rise to the now famous ``cooling flow problem'',
and represented a significant challenge for models of galaxy formation and evolution (e.g., \citealt{silk77,rees77}). 
Anecdotally, after the problem's emergence, a movement began within the community to rename ``cooling flow clusters'' 
to ``cool core clusters'' (with the same definition that $t_\mathrm{cool} < H_0^{-1}$), in an attempt to disassociate the {\it phenomenon} from the {\it model}, which
now suffered serious challenges\footnote{Some, however, still argue that the term ``cooling flow'' should not be abandoned, even if only for nostalgic reasons. See, for example, \citealt{soker10}, who points out that the Planetary Nebula community did not change terminology despite PNe being entirely unrelated to planets. The Author remains agnostic on the ``cool core'' vs.~``cooling flow'' issue, and finds either term equally catchy. }. 



There are two overriding implications of the cooling flow problem: 
\begin{itemize}

\item  The classical cooling flow model is either entirely wrong or severely 
overpredicts mass deposition rates by factors approaching 1000 (while remaining correct in principle), or

\item  Radiative losses of the ICM are replenished by a non-gravitational heating mechanism.

\end{itemize}


There are, of course, very simple scenarios in which the cooling flow model {\it could} be considered ``wrong'' (loosely speaking). 
Assuming the ICM to be optically thin, for example, could result in critical underestimation of the effects of transport 
coefficients like thermal conduction (which we will discuss later). What is not arguable, however, is that 
{\it the ICM is unequivocally cooling}, given the simple fact that we directly observe it losing energy through its thermally-driven X-ray radiation. This must, at least in part, support the second scenario wherein an external heating mechanism is at play. We review what is currently 
the most favored heating candidate below.  



















\subsection{Evidence in support of AGN feedback}


\citet{mcnamara07} have compiled an excellent review of the current evidence in support of AGN as the heating mechanisms in 
cool core clusters. We refer the reader to that far more comprehensive work, and merely summarize their conclusions here. 


\noindent {\bf Summary of current evidence supporting AGN feedback as the heating mechanism in CC clusters:} 

\begin{enumerate}


\item Consider the near 70\% detection rates of X-ray cavity systems in cool core clusters \citep{dunn05}, combined with the near 70\% detection rates
of radio emission in cool core clusters\footnote{It is important to note that these are not the ``same 70\%'', although there is significant overlap. The majority of CC clusters with detected X-ray cavities also possess extended radio emission.} \citep{burns90}. 
As ``icing on the cake'', the X-ray cavities often directly trace
the projected shapes of the extended radio sources, strongly suggesting that the former are created by the latter. 




\item If {\it not} regulated by a feedback loop, then heating rates {\it must} exceed cooling rates in order to ensure 
that no more than 10\% of gas can cool from the hot phase, and thereby match observations which turn up a dearth of the 
predicted cold reservoirs. However, heating rates that exceed cooling rates would raise the central entropy, eventually 
lengthening cooling times to the Hubble time. 
This would destroy the CC phase. Cool core clusters account for $>50$\% of the X-ray luminous 
cluster population below $z<0.4$, so the phase {\it must be long-lived}.
Of course, heating rates cannot be significantly lower than cooling rates either, or else the ``cooling flow problem'' 
would not exist to begin with. {\it The ICM must therefore be heated at a rate approximately of order its X-ray luminosity}.
A feedback loop, wherein the heating rate is directly coupled to the cooling rate, would be a simple explanation for this. 


\item Two obvious heating mechanisms which could establish a feedback loop with the cooling rate are (1) supernova feedback 
arising from star formation associated with the cooling flow, and (2) central AGN in the BCGs that are triggered by 
accretion reservoirs that stem from the cooling flow. Supernova feedback has been shown for many years to suffer crippling energetics issues, 
and has effectively been ruled out (as the dominant heating mechanism, at least, e.g., \citealt{voit05}). 



\item Nearly 90\% of energy radiated away in a cooling time must be replenished, and this must happen on timescales 
roughly as short as the {\it shortest} cooling time in order to account for the relative lack of observed mass sinks (star formation, 
cold gas reservoirs) in CC BCGs. 


\item Entropy profiles in CC clusters (which are generally power laws, \citealt{piffaretti05}) decay inwards toward the core, where
they flatten by levels consistent with the supposed amount of entropy injection by AGN ($\sim10$ keV cm$^2$, \citealt{mccarthy04,markmegan05}).  




\item AGN energy injection estimates based upon measurements of the ``$p \times dV$'' work associated with X-ray cavity 
excavation strongly correlates with X-ray luminosity (see Fig.~6 in \citealt{rafferty06}). This is {\it consistent with} (though not necessarily 
{\it suggestive of} \footnote{When making mention of any ``luminosity vs.~luminosity'' plot, 
one should at least bear in mind the words of Dr.~Robert ``Ski'' Antonucci, who best paraphrased the intentions of Robert Kennicutt:
\begin{quotation}
``{\it Kennicutt (1990) has extended the L(CO)-L(FIR) correlation for galaxies by adding points for a burning cigar, a Jeep Cherokee, the 1988 Yellowstone National Park 
forest fire, and Venus. I think the lesson is you get correlations in luminosity-luminosity plots (even without any sample completeness) just because 
big things have more of everything. There is probably a good correlation between the number of bookstores in a city and the number of bars, even 
with a complete sample, but no direct correlation is necessarily implied.''} -- \citet{antonucci99}
\end{quotation}Note that this should not be mistaken for criticism of the excellent (and convincing) work by \citet{rafferty06}}) 
the notion that the AGN heating rate is thermostatically controlled by the cooling rate, as would occur in a feedback loop. 
 

\item If one {\it averages} over both CC cluster population and lifetime, there is apparently {\it just enough} 
mean heating power from AGN ($1.01\times10^{45}$ erg s$^{-1}$) 
to balance mean cooling power ($6.45\times10^{44}$ erg s$^{-1}$, \citealt{rafferty06}). 



\end{enumerate}




\noindent {\bf Important qualifiers and caveats to the above lines of evidence} 


\begin{enumerate}

\item Only slightly more than half of CC clusters with {\it observed} X-ray cavities are associated with enough 
energy (again, based on $pV$ estimates) to actually offset inferred cooling rates for that specific cluster \citep{dunn06,mcnamara07}. This argument, 
however, is strongly tempered given (1) extremely strong biases against X-ray cavity detection frequency due to e.g., the highly variable exposure times / count rates in X-ray images from {\it Chandra} archival 
studies; and (2) the highly episodic nature of AGN (e.g., $pV$-estimated AGN injection rates do not account for cavities which may have been 
inflated in the past and since dissipated).   


\item 30\% of all CC clusters do not have detectable X-ray cavities \citep{dunn06}. This supposed evidence against AGN heating 
must of course be weighed with the caveats in point (1) above. To reiterate, the combination of (1) difficulties associated 
with detecting X-ray cavities and (2) the fact that AGN heating is almost certainly episodic (``on again, off again'') significantly 
diminishes (though does not completely eliminate) the power of this argument. We must await comprehensive studies of {\it deeper} X-ray 
imaging in order to make more definitive statements (to the contrary or otherwise). 


\item Some CC clusters with detected X-ray cavities lack extended radio emission. 



\item It is not clear if the X-ray cavity ``method'' is truly an accurate tracer of AGN power injection (though it's probably one 
of the best available means). The efficiency by which cavity enthalpy is converted to ICM heating is also 
somewhat uncertain (we discuss this below).   


\item Energy injection from AGN heating is very ``spatially discrete''. In other words, only the ICM in the vicinity of (1) X-ray 
cavities and (2) sound waves arising from the creation of these cavities, will be {\it mechanically} heated by the AGN. To quote 
Dr.~Megan Donahue (private communication), ``[this is] {\it like saying we in principle have enough drywall
to block the drafts without saying where the drywall is}''.  


\item Hybrid models which include both AGN feedback {\it and} thermal conduction (which may be more important 
in the outer regions of the cool core) have proven to be successful in predicting, to some degree, the observed radial temperature, 
density, and entropy profiles. In some cases, these hybrid models are {\it more} successful than models that include AGN feedback only \citep{ruszkowski02,brighenti03,voit05}. The net implication is that many mechanisms, in addition to feedback, may be working {\it in concert} 
to offset cooling. While this is almost certainly true to some degree, debate is ongoing as to whether or not one process 
clearly dominates over all others.  

\end{enumerate}


We return to point (7) above: If one takes an {\it average} over CC cluster population and lifetime, one finds that there is enough 
{\it average} heating power from AGN to offset {\it average} cooling power. The mean heating power quoted by \citet{rafferty06}
was calculated by taking $4pV \times t_\mathrm{buoy}$ for the observed cavities. Here,  $t_\mathrm{buoy}$ is the buoyant lifetime, 
one of three common estimates used to determine X-ray cavity ages. We discuss 
these below. 


\subsection{A qualitative summary of X-ray cavity inflation and age dating}

Here we outline a very simple schematic for the excavation of an X-ray cavity 
by a radio source. Although we discuss the details of this model in a confident tone, it is important 
to keep in mind that this is a vastly oversimplified ``cartoon'', and not necessarily an 
accurate representation of reality. 


In this simple schematic, X-ray cavity inflation is expected to be initially rapid as the nascent 
radio jet expands supersonically into the ambient medium. 
At this stage, cavities excavated by the jet are small, narrow, and younger than the local sound-crossing time, 
as the (possibly) shocked ISM gas is found only at the 
tip of the supersonic propagation front  \citep{heinz98,ensslin02,mcnamara07}. 
These ``infant'' cavities would only be associated with minor and spatially small depressions in the X-ray surface brightness, 
and would be difficult to detect, particularly given a large amount of intervening X-ray gas along the line of sight.  
The jet decelerates to transonic speeds as its ram pressure 
equalizes with the ambient ISM pressure, truncating the supersonic inflation stage, typically 
at distances $\lae 1$ kpc from the radio core.  
The cavity continues to expand subsonically and will eventually ``catch up'' with the now-decelerated jet on 
slightly more than a sound crossing time, at which point the buoyant force 
takes over as the dominant mechanism regulating cavity dynamics. 

% $F_b = V g \left(\rho_\mathrm{gas} - \rho_\mathrm{cav}\right)$ 


The buoyant bubble rises amid the ambient pressure gradient and expands adiabatically during 
its ascent to maintain pressure equilibrium with the surrounding gas. It eventually becomes 
large enough to cause a detectable, roughly circular deficit in X-ray surface brightness.  
At this stage, the cavity's 
maximum age is set by its terminal velocity in the medium, 
\begin{equation}
v_t \approx \sqrt{\frac{2gV}{SC}}
\end{equation}
which is set by the balance of the local buoyant and drag forces. Here, $V$ is the volume of the cavity, $S$ is its cross-sectional area, 
and $g$ is the local gravitational acceleration. The majority of X-ray cavities (including those in A2597) are cospatial with the BCG stellar isophotes, implying 
that they rise amid a potential dominated by the BCG. The local gravitational acceleration at radius $R$ from the center of the host galaxy may therefore be inferred from the stellar 
velocity dispersion $\sigma,
$\begin{equation}
g \simeq \frac{2\sigma^2}{R},
\end{equation}
assuming that the galaxy is an isothermal sphere. 
The terminal velocity $v_t$ of the cavity is always subsonic, as the Kepler speed $v_\mathrm{K} = \sqrt{gR}$ is of order the 
sound speed $c_s$ in the gas 
\begin{equation}
c_s=\sqrt{\frac{\gamma kT}{\mu m_\mathrm{H}}}, 
\end{equation}
where $\gamma\simeq5/3$ is the ratio of specific heats and $\mu\simeq0.62$ is the mean molecular weight, appropriate for fully ionized gas.   


Based on these simple assumptions, three estimates are generally used in age-dating X-ray cavities. 
The most simple of these is the sound crossing time $t_{c_s}$, 
\begin{equation}
t_{c_s} = \frac{R}{c_s} = R \sqrt{\frac{\mu m_\mathrm{H}}{\gamma kT}}, 
\end{equation}
which assumes a direct sonic rise of the bubble along the plane of the sky to its current projected radius $R$. 
As the initial stages of cavity inflation are thought to be supersonic, followed by subsonic buoyant rise, 
this admittedly simple approach may best reflect an ``average'' of the two stages. 
Alternatively, if initial inflation is a small fraction of the cavity's age, the buoyant rise time $t_\mathrm{buoy}$ may be used, 
which takes drag forces into account. 
Following the discussion of the terminal buoyant velocity above, $t_\mathrm{buoy}$ is given by 
\begin{equation}
t_\mathrm{buoy} \simeq \frac{R}{v_t} \simeq R \sqrt{\frac{SC}{2gV}}. 
\end{equation}
Finally, cavity age is constrained by the time it takes for gas to refill 
the bubble's ``wake'', which can be roughly estimated by the time 
it would take a bubble of radius $r$ to rise through its own projected diameter, 
\begin{equation}
t_\mathrm{refill} = 2 \sqrt{\frac{r}{g}}.
\label{equation:trefill} 
\end{equation}
In Chapter 2, we undertake this exercise for the X-ray cavities in Abell 2597. For more comprehensive 
work focusing on the age dating of cavities, see \citet{birzan04,rafferty06}, and \citet{mcnamara07} for 
an updated review.  



\subsection{A note on X-ray bubble longevity and magnetic field draping}

\label{section:bubbledisrupt}


As a purely hydrodynamic X-ray bubble buoyantly rises, it induces vortical motions in the ambient fluid, 
its leading edge sees the rapid growth of 
of Rayleigh-Taylor instability modes, and shear along the bubble walls gives rise to secondary Kelvin-Helmholtz 
instabilities (e.g., \citealt{soker02,robinson04,dursi08}, and references therein).
Together, these effects should disrupt an initially spherical, hydrodynamic X-ray bubble 
into a ``smoke ring'' on one bubble rise time,  $t_\mathrm{refill}$ (see Eqn.~\ref{equation:trefill}). 
Roughly circular, well defined X-ray cavities have been observed at clustercentric radii many times larger than their own radii, however, 
indicative of ages far older than one bubble rise time (if they are indeed inflated at the core, as
the AGN heating model suggests). 


Of course, a purely hydrodynamic description of X-ray bubbles is not adequate,
as magnetic fields play an important role in the hot ICM (see the discussion in
Section \ref{section:magneticfields}). Magnetohydrodynamic simulations of
bubbles have shown that the draping of magnetic fields over the leading edge of
the bubble can strongly suppress the growth of these instabilities, inhibiting
bubble disruption for long timescales \citep{robinson04,dursi08}. This is now one of the most 
favored scenarios accounting for the apparent longevity of X-ray cavities. For a more 
comprehensive discussion, see \citet{robinson04,mcnamara07,dursi08}, and references therein.  


 

\subsection{How heating by AGN feedback might work}
We again summarize some of the major points presented by \citet{mcnamara07}, and references therein. The energy associated 
with an X-ray cavity is equivalent to its enthalpy (``free energy'') $H$, which is merely the  $p \times dV$ work required to ``inflate''
the cavity, plus the thermal energy $E$ within the cavity. That is,  
\begin{equation}
H = E + pV = \frac{\gamma}{\gamma-1} pV,
\end{equation}
where $V$ is the cavity volume, $p$ is the pressure of the radio lobe which displaced the thermal gas, and $\gamma$ is the ratio 
of specific heats, dependent on the (largely unknown) contents of the cavity. If the bubble is filled by relativistic particles, 
we may take $\gamma=4/3$, in which case the cavity enthalpy would be $H=4 pV$. If the contents are magnetically dominated, 
then $\gamma = 2$ and $H= 2 pV$. A bubble filled with non-relativistic particles would be an intermediate case ($\gamma = 5/3$, $H=2.5 pV$).
Cavity enthalpy is therefore likely to be somewhere in the range of $2pV - 4pV$, regardless (to first order) of cavity contents.    

\citet{churazov02, reynolds02, birzan04} propose a simple mechanism wherein 
the enthalpy of a buoyantly rising cavity can be entirely dissipated  
as the displaced intracluster gas rushes to refill its wake. This results in the conversion of ICM potential energy $U$ to kinetic energy 
by a degree given by  
\begin{equation} 
\delta U = M g \delta R 
\end{equation}
where $M$ is the mass of the ICM that is displaced by the cavity as it rises through $\delta R$, and $g$ is the local gravitational 
acceleration, as before. Assuming $M=\rho V$, hydrostatic (so that $\rho g = - dp / dR$), and approximately isobaric local conditions (so the pressure gradient $\left[dp/dR\right]$ can be ignored, giving $\delta R = \delta p$), the potential energy dissipation is given by 
\begin{equation} 
\delta U =  V \left(\rho g\right) \delta R = -V \frac{dp}{dR} \delta R = -V \delta p. 
\end{equation}

The corresponding change in cavity enthalpy is simply given by the first law of thermodynamics,   
\begin{equation}
\delta H = T \delta S + V \delta p. 
\end{equation}
The cavity is assumed to be entirely adiabatic (non-radiative), so entropy $S$ remains constant and $\delta S=0$. 
We therefore see that
\begin{equation}
\delta H = V \delta p = - \delta U, 
\end{equation} 
meaning the amount of kinetic energy associated with the thermal gas rushing to refill the cavity  ($-\delta U$) 
wake is equal to the enthalpy (free energy) of the cavity. This is the general idea behind ``effervescent'' cavity 
heating models like those described by \citet{begelman01,ruszkowski02}. 
In Chapter 2, we explore energy budgets associated with this model in the context of the Abell 2597 X-ray cavities.  


Long-lived sound waves and weak shocks associated with cavity inflation 
are also expected to heat the ICM. 
By  injecting {\it potential} energy into the ICM (effectively the opposite of the cavity heating model), 
the waves decrease gas density and  therefore 
lower the cooling rate (which, if you will recall, is proportional to the square of the gas density). 
There is very strong observational evidence in the Perseus cluster (work by Fabian and collaborators) 
of sound waves driven during excavation of its cavities\footnote{Anecdotally, the evidence in Perseus is so 
strong that it enabled a rough estimate of the frequency of the sound waves: 3C~84 may effectively be ``playing'' a ``B-flat'' into Perseus at $\sim 60$ octaves below ``middle C''. This would  
be one of the lowest known ``musical notes'' in the Universe (A.~Fabian, private communication; see the press release from Fabian and collaborators). }. The waves are expected to dissipate into heat 
over longer timescales. Again, see \citet{mcnamara07} for a more comprehensive review of the effervescent cavity heating model. 


%\subsection{Alternative heating mechanisms}

%Many alternative heating mechanisms have been proposed. Some, as we will mention below, have been effectively 
%ruled out on (e.g.) energetics grounds. Others might feasibly (a) stand on their own or (b) work in conjunction 
%with AGN feedback in offsetting catastrophic ICM losses (for example conduction, \citealt{ruszkowski02,breghenti03}). 


%\subsubsection{{\bf Conduction}}

%The fact that the inctracluster gas is observed to cool via the emission of X-rays 
%means that cooler gas phases are embedded within hotter gas phases. 


\section{Brightest Cluster Galaxies in Cool Cores: Testing Cooling Flow and AGN Feedback Models}


This thesis (particularly the first half) is broadly related to the AGN feedback model as it pertains to star formation in brightest cluster galaxies. 
It is therefore important to briefly review the observed properties of BCGs in a relatively model-agnostic 
manner.  






\subsection{Observed properties of BCGs}

To again quote Dr.~Megan Donahue, brightest cluster galaxies are the ``trash heaps'' of the Universe. 
As they reside at the bottom of cluster potential wells, they are subject to extremely high relative merger rates at early times, gas-rich tidal stripping from nearby companions, 
and (if the cooling flow model is at least correct in principle), the accretion of cooling intracluster gas 
at late times. These factors conspire, yielding masses that can exceed $\sim 10^{12}$ \Msol\ 
and stellar halos that can extend hundreds of kpc into the cluster center, making BCGs  by {\it far} the largest and most luminous galaxies in the Universe (e.g., \citealt{sarazin88}). 


BCGs clearly stem from more extreme formation histories than do typical giant elliptical (gE) field galaxies (e.g., those galaxies not associated with a cluster), 
and so their properties differ in several important ways: 


\begin{itemize}


\item As stated above, BCGs as a class are generally much more massive and luminous than gEs by about an order of magnitude. 


\item Although more luminous, BCGs generally have lower central surface brightness than do gEs, 
and their outer stellar halos do not follow a typical  $\left(R/R_e\right)^{1/4}$ de Vaucouleurs' law (as gEs do). 

\item Optically selected BCGs {\it in non-cool core} clusters are {\it less} likely to harbor active central emission line nebulae than are gEs, by about a factor of two (10\% vs.~20\%, respectively).

\item Optically selected CC BCGs {\it in cool core clusters} are {\it far more} likely to harbor  active central emission line nebulae than are gEs, by more than a factor of two (45\% vs.~20\%, respectively, \citealt{best06,edwards07}). It follows that CC BCGs are more likely than non-CC BCGs to host these systems by a factor of four (45\% vs.~10\%, respectively). 

\item CC BCGs are more likely to host radio-loud AGN than are non-CC BCGs or gEs \citep{burns90}. 

\item Radio sources associated with CC BCGs are unusual as compared to radio-loud AGN as a whole: 
they are generally more compact ($<30$ kpc), have steep spectral indices at low 
frequencies ($\sim -2$), and are more ``blob-like'' in morphology \citep{odeabaum87,baum87}. 

\end{itemize}








\subsection{Origin of the Emission Line Nebulae and Star Formation in CC BCGs}
\label{section:bcgsincoolcores}

The clear connection between cool core clusters and the optical emission line nebulae in BCGs has been known for decades \citep{hu85,baum87,heckman89}, 
but consensus has never been reached on the {\it origin} of this phenomenon. 
Independent of the ``cooling flow problem'', it is now very clear that the emission line nebulae are not exclusively the $10^4$ K phase of a cooling flow, 
given the issues discussed at length in section 1.1 (e.g., H$\alpha$ luminosities, forbidden-to-Balmer ratios, etc.). The debate has instead focused 
upon whether or not the gas reservoirs from which the emission line nebulae and stars form have (a) condensed from a cooling flow or (b) are deposited through 
mergers.

 
Those who reject the cooling flow model (given its inability to reproduce
observations) generally favor some form of the latter argument. Their position
is naturally sound, as mergers are the primary agents of structure growth in
$\Lambda$CDM cosmology, and their ability to quickly deposit large gas
reservoirs into their merger products make them efficient triggers of star
formation and black hole growth periods (this would account for BCGs being so massive and luminous). 
Then again, the majority of galaxies in the central regions of clusters are generally quite gas-poor, possibly because
their gas contents are easily stripped (tidally and via ram pressure) during cluster assembly.  
In fact, field gEs, which should have been subject to a {\it lower} merger rate than BCGs at early times, 
are {\it more} likely than non-CC BCGs to host emission line nebulae and star formation amid gas-rich reservoirs (by a factor of two, see above). 




\begin{figure}
\plotone{cen_colormap.eps}
\caption[{\it HST} absorption map of the dust lane in NGC 4696]{Absorption map of the dust lane in 
NGC 4696, the brightest cluster galaxy at the center of the 
Centaurus cluster of galaxies. The map was made via division of {\it HST} 
WFC3 F160W and ACS F435W broadband images. NGC 4696 has a very low 
FUV-derived star formation rate, and may be in the process of a merger, as 
evidenced by the dynamically young dust structures. At the redshift of NGC 4696, 1\arcsec corresponds to 0.20 kpc. Figure by Grant Tremblay.}
\label{fig:chap3_centaurus_colormap}
\end{figure}



Alternatively, there is strong evidence that the emission line nebulae
and star formation arise from gas reservoirs which have condensed
from the ICM in cool cores.  While only 10\% of optically selected BCGs in
non-CC clusters possess central emission line nebulae
\citep{best06,edwards07}, this fraction spikes to  $>45\%$ for BCGs in cool
cores \citep{crawford99,edwards07}.  Of these, about half are known to exhibit
ongoing star formation (detected by a variety of means;
\citealt{johnstone87,mcnamara89,odea04,odea08,quillen08,odea10}).  
These pieces of circumstantial evidence are strongly supportive of 
some intrinsic connection between clusters with short central X-ray-derived cooling 
times and the presence of emission line nebulae and star formation in BCGs. 
The strongest pieces of direct evidence are: 
\begin{enumerate}

\item Star formation rates in CC BCGs strongly correlate with upper
limits on ICM cooling rates from X-ray spectroscopy \citep{mcnamara04,rafferty06,salome06,quillen08,odea08,odea10}.

\item Velocity structures of the cold gas in CC BCGs are consistent with expectations 
for gas that has cooled from an ambient hot, static atmosphere \citep{jaffe05,salome06}. 

\item The power of central radio sources in CC BCGs correlates with X-ray derived cooling times \citep{odeabaum87}.

\item The H$\alpha$ luminosity correlates with the radio luminosity in CC BCGs\footnote{Then again, this is true of radio galaxies in general \citep{baum88,baum89}. As mentioned previously, ``bigger is bigger'', so this is not necessarily surprising or indicative of an intrinsic connection.} \citep{heckman89}. 

\end{enumerate}

Note that, of course, the above luminosity-luminosity correlations must always be considered in 
the context of Footnote 8 above (``big things have more of everything''). Regardless, 
the tight cross-correlations and high statistical significance of the results very clearly 
support a scenario wherein (1) at least a significant fraction of gas in the BCG condenses from the ICM, 
(2) coalesces within the BCG, and (3) provides the reservoir from which emission line nebulae arise, from which 
stars form, and from which black holes grow in a mutually connected way. 
It is {\it very difficult} to imagine a purely coincidental scenario wherein (1) short ICM cooling times
and (2) the highly active nuclei of exclusively cool core BCGs are not intrinsically and directly related. 
If they are {\it not} directly related, then gas-rich tidal stripping or wet mergers would need to occur preferentially in CC BCGs in order 
to explain the near five-fold increase in star formation and emission line activity. It is not obvious 
why this would be the case, particularly given the relatively gas-poor companions of CC BCGs. 


There are direct observational tests that can be used to help discriminate 
between these two scenarios. 
Gas condensing from a cooling flow and being deposited through a merger 
are expected to have very distinct dynamical properties. 
Cooling gas from a static hot atmosphere should be associated with low net angular momentum, 
with a flow velocity that gradually increases from the systemic velocity at large radii, outside the BCG, 
to velocities of order a few hundred kilometers per second at smaller radii, with low levels of observed 
rotation. On the other hand,  gas stemming from mergers should possess high net angular momentum at all stages, 
and therefore be associated with rotation and high velocities at distant radii. 
The now-onlining Atacama Large Millimeter/submillimeter Array (ALMA), with its unprecedented sensitivity and spatial resolution at sub-mm wavelengths, will be 
critical in discriminating between these two scenarios for nearby cool core and non-cool core clusters. 






\subsection{Star Formation and a Ghost Ionization Mechanism in the Cold Reservoirs }

Despite the strong circumstantial connections discussed above, 
it is now clear that the emission line nebulae 
are not simply the $10^4$ K phase of a cooling flow. 
While the bulk of the ``baryon budget'' for  these phenomena
might be accounted for by condensation from the ambient hot reservoir, 
the cooling flow model in isolation 
seriously underpredicts the observed H$\alpha$ luminosities by $2-3$ orders of magnitude
in a pure radiative cooling scenario \citep[e.g.,][]{heckman89,voit97}. 
The nebulae are also
characterized by inverted forbidden-to-Balmer ratios,  
requiring 
models, independent of or in addition to cooling, which impart a great deal of heating per ionization event \citep{donahue91,voit97}. 




Star formation plays a critical but not exclusive role in regulating the
physics of both the warm and cold components of the ISM in CC BCGs.
Substantial young stellar components have been observed in
half of all CC BCGs, and these can often account for 
the ionizing photons required to power the observed luminosity of the H$\alpha$ 
nebulae and extended diffuse components of Ly$\alpha$ that are always observed
to be cospatial with the more clumpy and filamentary FUV continuum emission from O and B stars \citep{johnstone87,romanishin87,  mcnamara89,  mcnamara93,mcnamara04,mcnamara04b,hu92,crawford93,hansen95,allen95,smith97, cardiel98,hutchings00,oegerle01,mittaz01,odea04,hicks05,odea08,bildfell08,loubser09,pipino09,odea10,mcdonald10,mcdonald11,oonk11}.
However, stellar photoionization almost never correctly predicts the observed 
forbidden-to-Balmer ratios in the gas, and some emission line filaments 
are devoid of any observable star formation whatsoever (e.g., many of the NGC 1275 filaments, \citealt{conselice01}). In addition to star formation, another 
mechanism is needed to calibrate the ionization state of the gas. 
Proposed models have included AGN photoionization (long ruled out due to lack of radial gradients in ionization state), repressurizing and fast shocks \citep{cowie80,binette85,david88}, 
thermal conduction (e.g.,
\citealt{sparks89,narayan01,fabian02,voit08})  reconnection of  magnetic fields
\citep{soker90}, self-irradiation by gas from a cooling flow
\citep{voit90,donahue91} and turbulent mixing layers \citep{begelman90}, as
well as heating by convective instabilities (e.g., \citealt{chandran07}) and,
recently, cosmic rays (\citealt{ferland08,ferland09,donahue11,fabian11}; Mittal et al.~2011, in preparation). 












\section{AGN Feedback in Context}


The addition of ``anti-hierarchical'' quenching of star formation is clearly needed in
merger-driven hierarchical galaxy formation models in order to account for the decades-old ``over cooling problem'', wherein 
star formation is effectively catastrophic, grows along with the complexity of structure, and makes 
predictions that are effectively {\it opposite} to what is observed. Namely, this ``anti-hierarchical'' mechanism 
is needed to explain 
\begin{itemize}
\item The exponential turnover at the bright end of the galaxy luminosity function (e.g., \citealt{benson03})
\item The bimodality of galaxies in color-magnitude space (i.e., galaxies inhabit a ``blue cloud'' or a ``red sequence'', 
separated from one another by a very under-dense ``green valley''. Evolution from ``cloud-to-sequence'' is too rapid 
to be accounted for by passively evolving star formation rates. They must instead be actively quenched). 
\end{itemize}

The need for this anti-hierarchical growth mode can be broadly contextualized
as a need to explain the ``cosmic downsizing'' phenomenon, wherein the dominant contributors 
to the star formation rate density shifts from high mass galaxies to low mass galaxies with increasing cosmic time. 
The net implication is that more massive galaxies form at higher redshifts, and less massive galaxies form at lower redshifts --- this ``big-to-small'' behavior is not predicted in cosmological models based upon hierarchical, ``big-to-bigger'' growth.

 
Moreover, the $M_\mathrm{BH}-\sigma$ relation \citep{magorrian98,ferrarese00,gebhardt00}, if real, 
suggests that 
the growth of black holes and their host galaxy stellar components is not only {\it tightly coupled}, but also {\it self calibrating}. 
To first order, it seems that if the  $M_\mathrm{BH}-\sigma$ relation is intrinsic, it must be fostered 
by mutual growth amid gas coalescing in newly formed galaxies. But in order to explain the very tight calibration, 
there may also be a second order effect wherein this infalling gas triggers, for example, a feedback loop. 

The three most important pieces of evidence which point to AGN  feedback as a promising solution are: 

\begin{itemize}

\item the ``AGN cosmic downsizing phenomenon'', wherein lower luminosity AGN peak in  
comoving space density at lower redshifts than do higher luminosity AGN, which peak at $z\sim2$ \citep{giacconi02,cowie03,steffen03,hopkins07}. The 
fact that this observed phenomenon closely resembles the ``cosmic downsizing'' of star formation rate densities is probably 
not a coincidence. 

\item The onset of decline in star formation rates occurs at $z\sim2$ \citep{perezgonzalez07}, and  
the epoch of peak quasar activity is at  $z\sim2$ \citep{schmidt91,steffen03,hopkins07}. 

\item The Kormendy relation implies that all bright galaxies harbor central black holes and that 
these black holes must grow, suggesting that all bright galaxies go through active phases \citep{kormendy95,haehnelt93}.

\end{itemize}




The notion that just one solution -- AGN feedback -- may partly account for several major outstanding 
questions in modern astrophysics is attractive for obvious reasons, 
not the least of which is elegance and simplicity. Given the (literally!) universal implications of the above 
issues, it is clear that the study of BCGs in cool core clusters --- where we may {\it directly} observe the effects of AGN feedback --- are critical to our understanding of galaxy evolution as a whole. 


Although the involved microphysics is poorly understood, AGN feedback has now become a necessary 
component in models of galaxy evolution. 
Many models now include AGN powered by continuous or discrete, merger-driven episodes of black hole growth. 
While there are many varieties of modern simulations, consensus is mounting that ``quasar-mode feedback'' at early times  
and ``radio-mode feedback\footnote{The difference between these two terms is subtle, but important: in ``quasar-mode feedback'', powerful winds associated with a quasar 
quench star formation on global scales, accounting for the rapid ``blue-to-red'' evolution of galaxies. In ``radio-mode feedback'', radio-bright AGN outflows heat the ambient medium and suppress cooling flows by the mechanisms 
described in Section 1.2. This thesis focuses more on the radio-model model.}'' at late times yield results which match observations to a considerable degree, especially 
given the involved uncertainties and assumptions. 
In the so-called 
``quasar-mode'', wet mergers drive rapid black hole and stellar bulge growth between $2 < z < 3$, imprinting a persistent   
$M_\mathrm{BH}-\sigma$ relation that is fairly consistent with the observed relation. 
Eventually, the associated quasar winds quench star formation at $z\simeq2$ (corresponding with observations),  
marking the onset of more quiescent ``radio mode'' feedback epochs needed to suppress cooling flows at late times \citep{springel05,dimatteo05,sijacki06,croton06}.  
The most successful radio-mode models are episodic at rates corresponding to the average AGN duty cycle. 
For example, models by \citet{sijacki06} include effervescent, feedback-driven bubble injection every 10$^{8}$ yr, and show  
that bubble heating successfully prevents massive galaxies from growing too large. Not including the bubble heating 
results in unphysically massive galaxies at late times. Critically, while radio-mode feedback successfully suppresses star formation 
associated with cooling flows, it also allows {\it low} levels of star formation to persist beyond $z<1$, consistent with observations (see Chapters 2 and 3). 
The \citet{sijacki06} model also closely reproduces the entropy profiles of CC clusters, including the central 
flattening characteristic of kinetic energy injection from AGN.  









\section{A Note on Powerful Radio Galaxies}

Of course, the radio-mode AGN feedback model exists because powerful radio galaxies 
are commonly associated with BCGs in galaxy clusters (particularly in cool cores, \citealt{burns90}). 
Because of this, and because the later parts of this thesis deal specifically with radio galaxies in general, 
it is important to (very) briefly review our understanding of these sources. This is especially relevant as 
cluster environments and ICM cooling models may play critical roles in shaping two of the major 
observational dichotomies associated with radio galaxies, namely the Fanaroff-Riley and high- / low- excitation divides. We discuss these below, after a short review of the 
orientation-dependent unification paradigm. 







\subsubsection{{\bf Unified models of radio-loud AGN}}



%  Furthermore, FR~Is are  usually located  at the  center of
%massive  clusters \citep[see e.g.][for  a review]{owen96}.   On the
%other hand, at low redshifts, FR~IIs are generally found in regions of
%lower density, while some FR~II reside in richer groups or clusters at
%redshifts higher than $\sim 0.5$ \citep{zirbel97}.


Unification schemes for radio-loud and radio-quiet\footnote{Roughly 90\% of all AGN are considered 
``radio quiet'' (Seyfert galaxies and radio-quiet quasars). For the purposes of this thesis, 
we focus on the remaining 10\% of AGN considered ``radio-loud'' (radio galaxies, 
BL Lacs, and radio-loud quasars). For a review of the critically important radio-loud/radio-quiet dichotomy, see e.g., \citet{capetti10}.} AGN rely on   orientation-dependent obscuration to construct a singular model wherein all AGN, regardless of their many diverse properties, are intrinsically the same phenomenon viewed at different angles (e.g., \citealt{barthel89,antonucci93,urry95}).
In the  most basic, radio-loud scenario (the focus of this review),  an
accretion  region   about  a  central  supermassive   ($10^8  -  10^9$
\Msol) black  hole is  surrounded by an  equatorial obscuring
dusty torus. 
 The liberation  of  gravitational potential  energy from  the
accreting material  as it falls inward  toward the BH  powers not only
the characteristically  intense nuclear emission  associated with AGN,
but  also collimated  bipolar  outflows that  are  the progenitors  of $\sim100$  kpc (and  even  up to  Mpc)  scale radio  jets (which 
are of course the basis for the mechanical AGN feedback model). 
For lines of sight nearly  perpendicular to the radio jet axis, the 
dusty torus obscures the accretion region and the surrounding broad line 
region (BLR), while the more extended narrow-line region (NLR) remains visible. 
An observer 
with this line of sight would classify the object as a narrow-line radio galaxy (NLRG). 
Conversely, emission along a viewing angle nearly parallel to the jet axis 
would be dominated by relativistically boosted, non-thermal synchrotron processes in the jet,
and the object would be classified either as a radio-loud flat-spectrum quasar (QSO)  
or a BL Lac, depending on the presence or absence (respectively) of strong emission 
lines. Along intermediate viewing angles, wherein both the BLR and NLR remain visible, 
the observer sees a steep-spectrum, radio-loud QSO or a broad-line radio galaxy (BLRG) for higher 
and lower intrinsic nuclear luminosities, respectively. 



The model effectively implies that all type 2 AGN harbor hidden type 1 nuclei\footnote{Broad-line AGN are typically called ``type 1'', while narrow-line AGN are called ``type 2''. }, 
and the spectropolarimetric detection of highly polarized broad emission lines -- in some otherwise narrow-line radio galaxies -- has lent important supporting evidence in this regard (e.g., \citealt{robinson87,cimatti97,ogle97,cohen99,zakamska05}).
Stemming from this simple geometry-based scenario is the 
paradigmatic view that quasars and powerful radio galaxies 
intrinsically belong to the same population, viewed at different orientations. 
An extension of this is the view that low-power radio 
galaxies may be unified as the parent population of BL Lac objects, but this particular
hypothesis has since given way to mounting evidence to the contrary. Indeed, 
while elegant in its simplicity,  the orientation-dependent unification scheme has increasingly suffered from major 
unanswered questions -- and even outright problems -- as understanding of radio galaxies has evolved. 
Of course, this is not to say that the model is {\it wrong} (quite the contrary), but it clearly 
requires additional levels of complexity, for reasons we discuss below. 


\subsubsection{{\bf Current open problems with radio-loud unification models}}


Extended, $100$ kpc scale radio emission associated with radio galaxies has
long been classified into two morphological groups: lower power, edge-darkened
Fanaroff-Riley (hereafter FR) class I objects with (often) two ``plume-like''
jets, and higher power, edge-brightened FR II objects characterized by bright
terminal hotspots and the absence of an obvious counterjet \citep{fanaroff74}.
This dichotomy in morphology and luminosity is almost certainly intrinsic
(i.e., it is not a result of orientation), and the luminosity break separating
the two classes appears to depend on host galaxy brightness
\citep{owen94,ledlow96}.  FR I are generally lower redshift than are FR IIs,
which is not surprising given the tight redshift-luminosity correlation wherein
higher power sources will preferentially be selected at higher redshifts (i.e.,
the Malmquist bias arising from flux-limited samples).  
This should therefore not be mistaken as inference that FR~I are less numerous at 
high redshifts than FR~IIs; in fact, the opposite may be true (we explore
these issues in Chapter 5). Whether or not FR~Is
and FR~IIs are intrinsically distinct objects, or if they represent stages of
an evolutionary sequence, should therefore not be inferred from their redshift
distributions. 


The Fanaroff-Riley dichotomy is not well understood, and its better
characterization has been a major pursuit of the field for decades, as it
clearly stems from important jet physics.  An increasingly popular view is that
all initially propagating jets are launched relativistically, and FR Is stem
from the disruption of those jets over sub-kpc scales, slowing to possibly
trans-sonic speeds, likely through interaction with the ambient medium through
which they propagate (e.g., \citealt{bicknell95,laing08}).  In this case,
intrinsic differences in jet power would help to reconcile the Owen-Ledlow
divide between the high radio luminosities of FR IIs and the lower luminosities
of FR Is.  In fact, if all jets really are intrinsically FR IIs with varying
total powers (coupled, perhaps, to the accretion rate or mode), the Owen-Ledlow
divide may be a self-reinforcing phenomenon: higher power, intrinsically FR II
jets might better survive propagation through a dense ISM during the initial
stages of its launch and therefore {\it retain} their FR II morphology on 100
kpc scales. On the other hand,  lower power, a would-be FR II jet may become
disrupted through ISM mass loading, and eventually form a lower-power jet of FR
I morphology.  The Owen-Ledlow divide is more of a ``rough boundary'' than it
is a true divide, however, so some high luminosity FR I radio galaxies approach
the brightness of low-power FR II radio galaxies.  Problems with the above
interpretation therefore arise when one observes a high power FR I in a sparse
environment or a low-power FR II in a dense environment: if the former was
disrupted, why wasn't the latter?


Most evidence is supportive, however, as the vast majority of FR~Is are associated with cD ellipticals in clusters \citep[see e.g.][for  a review]{owen96}. 
FR~IIs, particularly at low redshifts, are almost invariably associated with Mpc environments 
that are far lower density than those for FR~Is. The contrary evidence 
arises mostly from the {\it small} population of FR~IIs that are found in the centers of 
richer groups or clusters at redshifts higher than $\sim 0.5$ \citep{zirbel97}.
Cosmological evolution may certainly be at play (e.g., \citealt{sadler07}). 
While consensus has not yet been met, the clear fact that FR~I reside in higher density environments 
is strongly supportive of the notion that it is the {\it environment} which 
gives rise to the Fanaroff-Riley dichotomy. Current quantitative models for mass entrainment 
in radio jets currently require an intrinsically FR II jet to be disrupted into an FR I 
on {\it sub-kpc} scales, however (e.g., \citealt{laing08}, and references therein). This is obviously problematic, as these scales 
are unrelated to cluster environment unless, for example, the ISM in the BCG is more dense because of a cooling flow (see e.g., Chapter 2, 
and the dynamically frustrated radio source associated with the BCG in Abell 2597). 


An overriding issue prohibiting better understanding arises because 
the  actual
{\it formation} of the jets is not  well understood, nor is the mechanism by
which they  remain tightly collimated  over such vast  distances.  Jet
formation models often include threading magnetic field lines, coupled
to the  accretion flow, through the  ergosphere of the Kerr  BH. A net
Poynting flux is  carried away by a relativistic  outflow aligned with
the BH spin  axis, thereby extracting BH spin  energy and providing an
outlet for excess angular momentum from the system (e.g., 
\citealt{blandford77,blandford82,punsly90,meier99,devilliers05}). 






Equally important as the Fanaroff-Riley dichotomy is the discovery that 
the nuclear optical emission associated with radio galaxies can be characterized 
as having either ``weak'' or ``strong'' emission lines, and that the divide 
appears to depend on radio luminosity. Weak-lined, low-excitation radio galaxies (LEGs) 
tend to have low radio power, while strong-lined, high-excitation galaxies (HEGs) have higher 
radio luminosities \citep{hine79,laing94,rector01}. As the Fanaroff-Riley dichotomy also traces a divide in radio power, 
many years have been spent investigating whether the FR I/FR II and HEG/LEG dichotomies 
are related. Indeed, most narrow-line FR IIs and (crucially) {\it some} FR Is harbor 
hidden quasars with associated broad lines, and those sources with a hidden quasar 
are typically HEGs, while those sources lacking a hidden quasar are typically LEGs. 


As this understanding was evolving, a confused and now largely inaccurate picture began to emerge: 
paper after paper began to associate FR Is, which are generally LEGs, as the parent population of BL Lacs (which are largely emission-line free). Similarly, high power FR IIs, which at the time 
were thought to generally associate with HEGs, became unified with flat-spectrum quasars. 
This view is almost certainly false, and some have argued that it achieved near-paradigm status
because the community continually focused on -- and even {\it assumed} -- that the FR I/FR II and 
HEG/LEG dichotomies were directly related (e.g., Hardcastle et al. 2009, and references therein).  
Indeed, the HEG/LEG divide does not trace the Fanaroff-Riley divide with a one-to-one correspondence. 
There is a significant population of FR II radio galaxies with LEG-type optical spectra, and there are sources 
with FR I radio morphology and high-excitation optical properties \citep{hine79,barthel94,laing94,jackson97,chiaberge02,hardcastle04,whysong04}. Examples include the famous broad-lined FR~I 
radio galaxy 3C 120, quasars with FR I morphology (e.g., \citealt{blundell01}), and BL Lacs with 
FR II morphology (e.g., \citealt{rector01}), though these outliers typically reside near the FR~I/FR~II luminosity break. 
In a way, the Fanaroff-Riley and HEG/LEG dichotomies have conspired together to vastly complicate
and even confuse the field of radio galaxies in the two decades since the (``official'') 
emergence of the orientation-dependent unification scheme\footnote{In truth, implying that \citet{urry95} marked the ``official'' emergence of the orientation-dependent unification scheme isn't quite 
fair: more than two decades of important preceding work lead to the notion that 
orientation plays a major role in unification. See, for example, \citet{antonucci84}.}. 


\subsubsection{{\bf A unifying connection between radio galaxies and cool core clusters?}}

Now, there is even an emerging notion that LEGs may {\it completely lack} the
fundamental components of the unification scheme, namely the obscuring torus
and radiatively efficient accretion disk
\citep{zirbel95,baum95,chiaberge02,whysong04,hardcastle06}.  Those works instead argue that the
X-ray to radio SEDs of such systems can be almost entirely reconciled with
emission from the base of the jet.   In such a scenario, the LEG/HEG dichotomy
may stem from different accretion modes, wherein HEGs arise from the accretion
of a cold gas reservoir stemming from a recent merger with, or tidal stripping
from, a gas-rich companion \citep{baldi08}.  Following the merger, the cold gas
settles into a symmetry plane of the host galaxy and falls inward toward the
nucleus at a rate dependent on the efficiency with which the gas sheds angular
momentum.  Eventually, the cold gas reaches the nucleus and forms the requisite
structures required in the orientation-dependent unification scheme, i.e.~the
cold dusty torus and the geometrically thin accretion disk.  On the other hand,
work by (e.g.) \citet{baum92,baum95} gave rise to the now widely argued notion
(e.g., \citealt{hardcastle07}, and references therein) that LEG are powered by
the ``hot-mode'' accretion of hot ($\sim 10^7$ K) gas stemming from cooling
flows in CC clusters.
Ignoring the outlier populations (for the moment), if we (dubiously) assume that in {\it general}, ``FR~I are LEG'' and ``FR~II are HEG'', 
a consistent picture arises wherein CC clusters play a fundamental role in shaping the two 
observational dichotomies (FR~I/FR~II and HEG/LEG) in radio galaxies. 

 

In this scenario,  
cooling from the hot phase accretes onto the central BH, triggering the AGN, which in turn 
re-heats any cold gas that may have accumulated (both radiatively and by the mechanical mechanisms discussed above). 
This effectively prevents the gas from cooling on any significant scale, 
inhibiting coalescence of the material into the dusty torus and thin accretion disk required by the unification scheme. 
This picture is not complete, and the hot-mode accretion model is thus far incapable
of sufficiently powering most narrow-line radio galaxies \citep{hardcastle07}. 
Moreover, the argument by Chiaberge and collaborators that FR~I may lack an obscuring torus operates 
under the assumption that emission from the base of the jet stems from a size scale {\it smaller} than 
any obscuring torus that may or may not be present. If the optical emission associated with the base of the synchrotron jet is from a region larger than 
the size of the torus, then its presence or absence cannot be directly inferred.
It is also worth noting that settled, kpc-scale dusty disks and lanes {\it are} observed in FR~I 
radio galaxy CC BCG hosts. Moreover, the FR~Is are generally more dusty than FR~II, and their dust distributions 
generally appear more ``dynamically settled'' (e.g., coherent disks as opposed to filamentary, wispy lanes, \citealt{dekoff00,tremblay07}). 
While these kpc scale dust structures should not be confused for obscuring tori on far smaller scales, 
it is at least worth questioning why the former structures could form but not the latter. 
Clearly there is at least {\it some} important connection uniting the 
Fanaroff-Riley and HEG/LEG dichotomies with cluster environments (or lack thereof), 
but the strength of this connection is, as of yet, entirely unknown. This is certainly 
an instance where the importance of outliers (i.e., FR~II LEGs, FR~I HEGs and FR~IIs in rich clusters) cannot be downplayed or ignored. 




\section{In this Thesis}


We have so far provided a very broad contextual summary for the issues related
to this doctoral thesis. Much of this is background material, of course, and
only a very small fraction of the issues discussed above will be directly
addressed in the forthcoming chapters. Before summarizing the pages to come, it is important to keep in mind that {\it in order to know an 
answer,  we must first know the question}. This thesis will not
{\it answer}, either fully or in part, any questions pertaining to 
cool cores, AGN feedback, or CC BCGs. Rather, it is our hope that 
the observational results we present will further our understanding, 
at least in a small way, of some of the major {\it questions} related 
to these issues. Below we summarize a subset of these questions, 
and describe how their better understanding may be brought about 
by the results presented in this thesis.  \\



\noindent {\bf The questions this thesis will address}

\begin{itemize}


\item  {\bf What role does AGN feedback play in regulating star formation and 
the entropy of the ICM?} 
What might be learned by comparing the morphology and magnitude
of compact star forming regions in CC BCGs with the energy and timescale 
budgets associated with effervescent AGN feedback models? 
In {\bf Chapter 2} we will present 
an in-depth, unified study 
considering the energy, mass, 
and timescale budgets of both the hot and warm/cold gas phases
in the BCG of the cool core cluster Abell 2597. This canonical, well-studied
source exhibits both low levels of ongoing star formation and 
signatures of episodic, effervescent AGN feedback. 
Using new {\it Chandra} and {\it HST} observations, we will provide 
new insights into the effervescent AGN feedback model, and the role 
it may play in regulating star formation and heating the ambient ICM/ISM. 


\item {\bf What is the role of star formation in 
regulating the physics of the warm and cold ISM phases in CC BCGs?}
In {\bf Chapter 3} we will present {\it HST} FUV imaging for a sample of 
seven CC BCGs selected on the basis of an IR excess thought to be associated 
with elevated levels of star formation. We will show that these young 
stellar populations play a critical, even dominant role in 
providing the reservoir of ionizing photons needed to power the 
emission line nebulae in these systems. A comparison with radio data 
and results from the literature may reveal clues into the behavior of systems
experiencing a low level of feedback from the AGN, which may allow for
increased residual condensation from an ambient hot atmosphere, accounting 
for the heightened star formation rates. 


\item {\bf How can we disentangle the roles played by cooling flows and 
mergers in depositing the gas reservoirs which fuel episodes of star 
formation and AGN activity?} What might we learn by comparing 
Abell 2597, associated with a moderately strong cooling flow, with 
a radio galaxy not associated with a cluster, but also exhibiting 
evidence of episodic star formation and AGN activity? 
In {\bf Chapter 4} we will present a multiwavelength study of the giant 
radio galaxy 3C~236, one such source. Clearly, a cooling flow 
cannot be responsible for triggering star formation and AGN activity in this object. Might the differences between A2597 and 3C~236 yield new insights into 
hot and cold accretion mechanisms?  


\item {\bf How might we find assembling protoclusters 
at high redshift, so as to study the cosmic evolution of (e.g.) the cool 
core phenomenon? } Galaxy clusters assemble at redshifts $1 < z < 2$, though 
to date, very few have been detected in this redshift range because 
of the difficulties associated with extending galaxy cluster selection 
methods to these depths. Understanding protoclusters at high redshift 
will be critical to understanding clusters in the local Universe, 
particularly in terms of the cool core/non-cool core dichotomy. 
Studying the cosmic evolution of FR~I radio galaxies, many of which are embedded
in clusters, may lend new insights into these issues. 
In {\bf Chapter 5} we
will broaden the context of this thesis with a search for high redshift FR~I radio
galaxies that may act as observable ``beacons'' for assembling protoclusters. 
 


\end{itemize}

\noindent Finally, in {\bf Chapter 6}, we will revisit the questions 
posed above, and assess what we have learned. 





\begin{figure}

\plotone{m87_xray_radio.eps}
\caption[The cool core cluster M87 ({\it Chandra} X-ray, VLA Radio)]{Radio (pink) and X-ray (blue) composite 
of M87, the CC BCG in Virgo. See \citet{harris06,forman07,sparks09,batcheldor10} for some recent papers on this canonical radio galaxy (Credit: X-ray: NASA/CXC/KIPAC/N. Werner, E. Million et al; Radio: NRAO/AUI/NSF/F.~Owen).}
\end{figure}


 
